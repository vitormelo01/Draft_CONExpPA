\documentclass[../Main.tex]{subfiles}

\begin{document}

Nursing Home CON laws impose bureaucratic costs to nursing home providers by requiring those planning on opening a new nursing home or expanding a current facility to first show to a regulatory body that their region needs the service. \footnote{An alternative way to think about nursing home CON laws, especially the most restrictive regulations involving a moratorium, is that they impose a restriction on the quantity of nursing home services/beds. This restriction would result in a higher price, a lower quantity, and excess demand, the same outcome as modelling the regulations as a government-imposed decrease in supply.} Propositions (1), (2), and (3) predict that repealing nursing home CON will lead to an increase in nursing home quality, an increase in nursing home services, and an increase in total nursing home expenditures and Medicaid nursing home expenditures respectively. It follows that we would expect the repeal of nursing home CON in PA, IN, and ND to increase quantity of nursing homes and nursing home beds, as well as an increase in total nursing home expenditures and Medicaid nursing home expenditures in all states. Given the lack of data of historical data on the quality of nursing homes, only the impact on quantity and expenditures are analyzed.  \\ 
\indent Our empirical findings are consistent with the predictions of our model in all instances with strong pre-treatment parallel trends. Figures \ref{fig:q_nursing_homes_trends} and \ref{fig:q_nursing_home_beds_trends} show trends in the quantity of nursing homes and nursing home beds per 100,000, respectively, in PA, IN, and ND and their corresponding synthetic controls. The quantity of nursing homes and beds per 100,000 in PA, IN, and ND and their corresponding synthetic controls are virtually identical up until they dropped nursing home CON regulations, at which point there is a clear divergence with the quantities in the states that dropped nursing home CON increasing relative to their synthetic controls. Recall from Section \ref{empirical_strategy} that we only used the first half of the pre-intervention years to form the synthetic matches using our outcome variable, and reserved the second half for out-of-sample validation. Despite only matching on the first half of the pre-intervention period, the outcomes in each state match their synthetic control extremely well in the second half of the pre-intervention period as well, instilling confidence that the synthetic controls replicate well the respective counterfactual trajectories of PA, IN, and ND in the post-intervention period.\\
\indent The year-specific effects of dropping NH-CON regulations on the quantity of nursing homes and beds, $\hat{\alpha}_{it}$ from equation (\ref{eq:year_spec_effect}), are shown graphically in Figures \ref{fig:q_nursing_homes_gaps} and \ref{fig:q_nursing_home_beds_gaps}, respectively. Also shown in these figures are the year-specific placebo effects for the states in the donor pool of potential control states, allowing for a visual inspection of how large the year-specific effects are in the treated states in comparison to the distribution of placebo effects. Our findings indicate that dropping NH-CON caused an immediate increase in the quantity of nursing homes per 100,000 in all three states with a subsequent gradual increase in PA and IN, and levelling off in ND. While there is less of an initial immediate increase, especially in PA and IN, a similar pattern emerges for the quantity of nursing home beds as shown in Figure \ref{fig:q_nursing_home_beds_gaps}. The year-specific effects in the treated states, especially PA and IN, appear to be relatively large when compared to the the distribution of year-specific placebo effects.\\
\indent Columns (3) and (4) of Table \ref{tab:ave_results} report the average annual effects, $\bar{\alpha}_1$ from equation (\ref{eq:ave_effect}), of dropping nursing home CON regulations on the quantity of nursing homes and beds per 100,000, respectively. Dropping nursing home CON caused an increase of 0.37, 0.28, and 0.21 nursing homes per 100,000 in PA, IN, and ND, respectively. The exact p-values corresponding to these estimates are 0.083 for PA, 0.028 for IN, and 0.472 for ND, suggesting that only the estimates in PA and IN are statistically significant at any reasonable significance level. These estimated effects are quite large when compared to the average quantity of nursing homes per 100,000 over the pre-intervention period. For example, an increase of 0.37 and 0.28 nursing homes in PA and IN is an increase of about 58.7 and 44.4 percent relative to their pre-intervention averages of 0.63. We find a similar positive effect on the quantity of nursing home beds per 100,000 in all three states, with the effect being largest and statistically significant in IN.\\
\indent Figures \ref{fig:tot_exp_trends} and \ref{fig:medicaid_exp_trends} show trends in total government nursing home and Medicaid expenditure per capita, respectively. Similar to the quantity trends, we find that our synthetic controls do a great job of matching PA despite only matching in the first half of the pre-intervention period, but do not match ID and ND as well. There's a clear divergence in total and Medicaid expenditure per capita after nursing home CON is dropped in PA, with total expenditure growing much faster in PA than in its synthetic control, and Medicaid expenditure increasing dramatically, especially initially, while Medicaid expenditure would have remained relatively stagnant had it not dropped NH-CON. Figures \ref{fig:tot_exp_gaps} and \ref{fig:medicaid_exp_gaps}, which show the year-specific effects in PA, IN, and ND and the placebo effects of dropping NH-CON on total and Medicaid nursing home expenditure, tell  the same story, and suggest that total and Medicaid expenditure increased dramatically in PA relative to the distribution of placebo effects, but the impact on ID and ND are insignificant.\\
\indent Columns (1) and (2) of Table \ref{tab:ave_results} report the average annual effects of dropping Nursing home CON on total and Medicaid nursing home expenditure per capita. Interestingly, the estimates of the effect on Medicaid expenditure are negative in both IN and ND, and also on total expenditure in IN. The p-values associated with the effect on Medicaid expenditures in IN and ND is very large suggesting no statistically significant effect of dropping nursing home CON on total or Medicaid expenditure in IN or ND. In contrast, we find large and statistically significant effects of dropping NH-CON on total and Medicaid expenditure per capita in PA. We estimate that dropping nursing home CON caused total nursing home annual expenditure to increase by \$123.57 per capita, an increase of about 35.3 percent relative to the average pre-intervention total expenditure of \$361 per capita. Most of this increase in total nursing home expenditure was in the form of increased nursing home Medicaid expenditure. We estimate that dropping NH-CON caused nursing home Medicaid expenditure to increase by \$104.17 per capita, an increase of about 71.6 percent relative to the average pre-intervention nursing home Medicaid expenditure of \$145.56 per capita.

\end{document}