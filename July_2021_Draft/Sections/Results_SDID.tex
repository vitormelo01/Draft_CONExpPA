\documentclass[../Main.tex]{subfiles}

\begin{document}

Nursing home CON laws impose bureaucratic costs to nursing home providers by requiring those planning on opening a new nursing home, expanding a current facility, or offering new beds to first show to a regulatory body that their region needs the service. Propositions \ref{prop1}, \ref{prop2}, and \ref{prop3} predict that repealing nursing home CON will lead to an increase in nursing home quality, an increase in the quantity of nursing home services, and an increase in Medicaid nursing home expenditures, respectively. Based on the results of the model, therefore, we expect the repeal of nursing home CON in PA, IN, and ND to result in an increase in Medicaid nursing home expenditure per capita, as well as an increased quantity of nursing home beds, nursing homes, and specialized care beds (our measures of access to and quality of nursing home services) in all three states. Our empirical findings, the details of which are found in the next two subsections, are consistent with these predictions of our model in all instances of statistical significance, and where pre-treatment trends are reliably parallel.\\
\indent We summarize our main results in Tables \ref{tab:ave_results_med_exp_nobord_nocov} through \ref{tab:ave_results_q_specbeds_nobord_nocov} and in Figures \ref{fig:med_exp_plots_pa} through \ref{fig:q_specbeds_plots_in}. Tables \ref{tab:ave_results_med_exp_nobord_nocov} through \ref{tab:ave_results_q_specbeds_nobord_nocov} report the average estimated effect, $\hat{\tau}$ from equation (\ref{eq:ave_effect_deltas}), of dropping nursing home CON regulations on each outcome variable of interest in PA (Panel A), IN (Panel B), and ND (Panel C) using the DID (column one), SC (column two), and SDID (column three) estimators. Also reported are standard errors and 95\% confidence intervals using the placebo variance estimation approach outlined in Section \ref{sdid_inference}.\\
\indent Figures \ref{fig:med_exp_plots_pa} through \ref{fig:q_specbeds_plots_in} show a series of plots for each outcome variable and treated state that provide insights into these average effects, pre-trends, and our placebo variance estimation approach to conducting inference outlined in Section \ref{sdid_inference}. More specifically, the top row of plots in each figure, which we refer to as the ``trend plots,'' show trends in the outcome variable over time for the treated state and for its corresponding weighted average of controls states, with the time weights ($\hat{\lambda}$) used to average pre-treatment time periods at the bottom of the plots. The curved arrows in each of the trend plots represent the estimated average treatment effects that are reported in Tables \ref{tab:ave_results_med_exp_nobord_nocov} through \ref{tab:ave_results_q_specbeds_nobord_nocov}.\footnote{Recall that the SDID estimator re-weights the untreated control states with the goal of making their time trends as parallel as possible, but not necessarily identical, to the treated state in the pre-treatment period, and then applies a DID analysis to this re-weighted panel. Moreover, because of the time weights, the SDID estimator only focuses on a subset of the pre-treatment time periods when carrying out this last step. These time periods are selected so that the weighted average of historical outcomes predict average treatment period outcomes for control states, up to a constant.} The middle row of plots are what we call ``spaghetti plots,'' which show the year-specific difference in the outcome variable between the treated state and its corresponding weighted average of control states for each of our estimators. The thick blue line shows these gaps for the actual treated state, and the thin pink lines show these gaps for each of the placebo control states used in the placebo variance estimation procedure outlined in Algorithm \ref{alg:two}. To facilitate a more meaningful comparison in how these gaps evolve over time, as well as a good way to visually assess pre-treatment parallel trends, we make the gaps in the DID and SDID plots relative to their value in the year prior to the treated state dropping nursing home CON. We do not do this for the SC plot because the SC weights are chosen to match the treated state's actual levels, as opposed to making the trends just parallel (see Section \ref{did_sc_comp}). Not normalizing the gaps for the SC plot allows for a better assessment of the pre-treatment match between the actual treated state, or the placebo ``treated'' state, and its respective synthetic control. The bottom row of plots, which we refer to as ``placebo distribution plots,'' show the distribution of placebo estimates ($\hat{\tau}^{(b)}$ from Algorithm \ref{alg:two}), with the mean of the placebo estimates and the estimated effect for the actual treated state indicated by the gray and blue vertical lines, respectively.

\subsection{Medicaid and Total Nursing Home Expenditure}

The magnet hypothesis developed in Proposition \ref{prop2} implies that the repeal of nursing home CON in a given state will make the state a type of magnet for nursing home consumers from other states, and perhaps for consumers who would otherwise leave the state. Because a large majority of nursing home consumers are covered by government programs (primarily Medicaid) where prices are relatively inflexible, this predicted increase in the quantity of nursing home services would be expected to result in an increase in Medicaid and total expenditures on nursing home services (Proposition \ref{prop3}).\\ 
\indent Tables \ref{tab:ave_results_med_exp_nobord_nocov} and \ref{tab:ave_results_tot_exp_nobord_nocov} report the average estimated effect of dropping nursing home CON regulations on per capita Medicaid and total nursing home expenditure, respectively. Our preferred SDID estimates suggest that, in PA, dropping nursing home CON caused an increase of about \$116 in total per capita nursing home expenditure, a 32 percent increase relative to the pre-intervention mean per capita expenditure of \$361.02. While all point estimates for IN and ND in Panels B and C are positive (with the exception of the SC estimate for IN), none are statistically different from zero. The point estimates in Panel A of Table \ref{tab:ave_results_med_exp_nobord_nocov} and appendix Table \ref{tab:ave_results_medicare_exp_nobord_nocon} (where we consider Medicare nursing home expenditure per capita as the outcome variable) suggest that the increase in total per capita expenditure in PA is being driven almost entirely by an increase in Medicaid spending. We find no effect of repealing nursing home CON on Medicare expenditure in PA. In contrast, according to our SDID estimates in Table \ref{tab:ave_results_med_exp_nobord_nocov}, dropping nursing home CON caused per capita Medicaid expenditure to increase by \$101.73 and \$48.76 in PA and IN, respectively. When compared to their pre-intervention average values, these estimated effects represent increases in per capita nursing home Medicaid spending of about 70\% in PA and 38\% in IN. Considering that nursing home services have historically represented over a third of all Medicaid expenditures \citep{wiener1999controlling}, these positive effects on per capita Medicaid expenditure are not only statistically significant, but economically meaningful as well. \\
\indent Figures \ref{fig:med_exp_plots_pa} -- \ref{fig:tot_exp_plots_nd} show for each treated state, the trend plots, spaghetti plots, and placebo distribution plots for Medicaid and total nursing home expenditure per capita. One consequential takeaway from these figures, which is generally true for all outcomes we consider, is that the unit weights used in SC and SDID estimation significantly improve pre-treatment similarities in the levels and trends of the treated states and their synthetic controls, facilitating less biased causal effects of repealing nursing home CON regulations. Notice in Panel A of Tables \ref{tab:ave_results_med_exp_nobord_nocov} and \ref{tab:ave_results_tot_exp_nobord_nocov}, for example, that the DID estimates for PA are relatively larger than the corresponding SC and SDID estimates. Figures \ref{fig:med_exp_plots_pa} and \ref{fig:tot_exp_plots_pa} illustrate that this upward bias is likely a result of non-parallel trends between PA and its corresponding DID control group. In contrast, there is strong evidence of parallel pre-trends between PA and its corresponding weighted control group using the SC and SDID estimators, instilling confidence that the SC and SDID control groups represent a good counterfactual trajectory for PA. Figures \ref{fig:med_exp_plots_pa} and \ref{fig:tot_exp_plots_pa} also show that for the SC and SDID estimators, the estimated effects on Medicaid and total expenditure per capita in PA are larger than every one of the SC and SDID placebo estimates.\\
\indent Our estimates in Panel C of Table \ref{tab:ave_results_med_exp_nobord_nocov} are all negative, suggesting that dropping nursing home CON in ND may have actually decreased Medicaid expenditure. However, an inspection of Figures \ref{fig:med_exp_plots_nd} and \ref{fig:tot_exp_plots_nd} reveal that these point estimates are unreliable given the obvious lack of pre-treatment parallel trends. In other words, given the spike and subsequent fall in Medicaid expenditure per capita in ND prior to repealing nursing home CON, the control groups, even with optimal SDID weights, fail to provide a reasonable counterfactual trajectory for ND. \\
\indent We also analyze whether and to what extent the repeal of nursing home CON affected Medicaid hospital and home care expenditure. If the estimated increase in Medicaid nursing home expenditure is in fact being caused by the repeal of nursing home CON, as opposed to some other state policy or factor, then we would not expect to see a corresponding increase in Medicaid hospital and home care expenditure. Tables \ref{tab:ave_results_med_hos_exp_nobord_nocon} and \ref{tab:ave_results_med_homecare_exp_nobord_nocon} in the online appendix show the average estimated effects of repealing nursing home CON on Medicaid hospital and home care expenditure per capita. We find no significant effects on these outcomes in any of the treated states, regardless of the estimator we use.


\subsection{Quantity of Nursing Home Beds, Nursing Homes, and Specialized Beds}

\indent In light of our findings of increased Medicaid and total nursing home expenditure, if prices are relatively inflexible, then we would expect to see an increase in the quantity of nursing home services as a result of repealing nursing home CON as well, especially in PA where the effects on expenditure are the largest. Our model Propositions \ref{prop1} and \ref{prop2} also predict an increase in both the quality and quantity of nursing home services, which we more formally test by analyzing how repealing nursing home CON affected the quantity of nursing home beds, nursing homes, and specialized beds per 100,000.\\
\indent Table \ref{tab:ave_results_q_nhb_nobord_nocov} reports the average estimated effects of dropping nursing home CON regulations on the quantity of nursing home beds per 100,000 for each of our three treated states and estimators. According to our preferred SDID estimates, repealing nursing home CON caused an increase of 14.53, 6.94, and 5.17 nursing home beds per 100,000 in PA, IN, and ND, respectively. Relative to their pre-intervention averages, the estimates for PA and IN are again quite large, representing increases of about 51 and 41 percent, respectively. Figures \ref{fig:q_certbeds_plots_pa}, \ref{fig:q_certbeds_plots_in}, and \ref{fig:q_certbeds_plots_nd} show the corresponding trend plots, spaghetti plots, and placebo distribution plots for the quantity of nursing home beds per 100,000 in PA, IN, and ND. Figure \ref{fig:q_certbeds_plots_in} illustrates another example of non-parallel pre-trends using the traditional DID estimator, and explains why the DID point estimate for IN is almost double the SC and SDID estimates. The SC and SDID estimates are similar for PA and IN, and Figures \ref{fig:q_certbeds_plots_pa} and \ref{fig:q_certbeds_plots_in} provide evidence of the parallel trends assumption being satisfied when weighting the control group using these methods. The spaghetti plots in Figures \ref{fig:q_certbeds_plots_pa}, \ref{fig:q_certbeds_plots_in}, and \ref{fig:q_certbeds_plots_nd} also illustrate that there is considerable noise in the placebo estimates, being in part driven by several extremely large placebo estimates.\\
\indent \citet{abadie2010synthetic} discuss removing placebo units that, in the context of SC estimation, do not have a reasonably good fit in terms of pre-period mean-squared prediction error compared to that of the treated unit. This same idea can be applied to the SDID estimation. We follow \citet{abadie2010synthetic} in dropping Washington DC from all of our analyses because we could not find anything remotely close to a good control group (in terms of pre-treatment levels and parallel trends) for Washington DC. While it is clear that some of the other placebo states had non-parallel pre-trends and a relatively poor fit in both SC and SDID analyses of the quantity of nursing home beds (and nursing homes), we chose to include all control states (with the exception of Washington DC) to be conservative and transparent, despite the resultant larger estimated variance and standard errors.\\
\indent A similar story emerges when we consider an alternative measure of the quantity of nursing home services -- the quantity of nursing homes per 100,000. Table \ref{tab:ave_results_q_nh_nobord_nocov} reports the average estimated effect of dropping nursing home CON regulations on this outcome. All point estimates are again positive, regardless of the treated state or estimator. According to our preferred SDID estimates, repealing nursing home CON caused an increase of 0.44, 0.25, and 0.28 nursing homes per 100,000 in PA, IN, and ND, respectively. These estimated effects are again statistically significant for PA and IN only, and are quite large when compared to the average quantity of nursing homes per 100,000 over the pre-intervention period. For example, an increase of 0.44 nursing homes per 100,000 in PA is an increase of about 70 percent relative to its pre-intervention average of 0.63. Thus, these results show that like Medicaid expenditure, the effect of repealing nursing home CON on the quantity of nursing homes is not only consistent with our model, but also substantial and economically meaningful, again most notably in PA. \\
\indent To the extent that newly produced facilities and beds are of relatively higher quality, the positive effect of repealing nursing home CON on nursing home beds and facilities in PA and IN provide suggestive evidence themselves that the quality of nursing home services also increased. We explore how the law changes affect nursing home quality further by analysing their effect on the quantity of specialized beds per 100,000, which have been shown to be correlated with higher quality of service in general \citep{grabowski2010quality}. Table \ref{tab:ave_results_q_specbeds_nobord_nocov} and Figures \ref{fig:q_specbeds_plots_pa} and \ref{fig:q_specbeds_plots_in} report our results for PA and IN.\footnote{We are not able to analyze the effect of repealing nursing home CON on the quantity of specialized beds in ND because there were no specialized beds in ND during the pre-treatment period in the data.} We find large positive effects of repealing nursing home CON in both PA and IN, though the estimates are only statistically significant for PA. More specifically, we estimate that repealing nursing home CON in PA caused an increase of about 2.26 specialized beds per 100,000 (the SC and DID estimates are very similar in this case). This is a massive increase relative to PA's pre-period mean specialized beds per 100,000 of only 0.33. Figure \ref{fig:q_specbeds_plots_pa} shows that the effect of repealing the law on the quantity of specialized beds is not homogeneous over time in PA. Rather, it appears there was an especially large initial increase in the quantity of specialized beds in the years immediately following the repeal of nursing home CON. We also see evidence in Figure \ref{fig:q_specbeds_plots_pa} of parallel pre-trends with each of the estimators, lending credence to each control group's ability to represent the counterfactual trajectory of PA.\\
\indent Considering our empirical results as a whole, we find evidence consistent with the predictions of our model in all cases of statistical significance, and where pre-trends are parallel. More specifically, we find that repealing nursing home CON causes an increase in the quality and quantity of nursing home services (Propositions \ref{prop1} and \ref{prop2}), as well as an increase in total and Medicaid nursing home expenditure (Proposition \ref{prop3}), predominantly in PA and IN.



\end{document}