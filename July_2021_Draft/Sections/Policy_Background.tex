\documentclass[../Main.tex]{subfiles}

\begin{document}

CON laws were first introduced in 1964 by the state of New York and became widely popular among US states in the 1970s. Due to The National Health Planning and Resources Development Act of 1974 (P.L. 93–641), which heavily incentivized states to implement CON regulations, all but one state (Louisiana) had implemented CON laws in some of their health sectors by 1980. This federal push for CON regulations was reversed in the early-1980s (P.L. 99–660, Title VII). Ten states repealed their nursing home CON laws soon after the reversal of this federal push in the early 1980s, others (PA, IN, and ND) repealed their CON laws in the 1990s, and New Hampshire repealed its CON laws in 2016. All other states, as well as Louisiana (adopted nursing home CON in 1991), Connecticut (adopted nursing home CON in 1993) and Indiana (which reinstated Nursing Home CON in 2019), currently have nursing home CON laws. This paper analyses the impact of the CON repeal in PA, IN, and ND because these are the only repeals that allow for sufficient pre-treatment and post-treatment time, as well as potential control states.\footnote{Louisiana and Connecticut allow for sufficient pre-treatment and post-treatment time but there are no comparison units since these are the only states that did not have nursing home CON in the 1980s.} \\
\indent Past research has analyzed the correlation between nursing home CON and expenditures, as well as quantity of nursing home services. Multiple studies have found that the presence of nursing home CON laws are associated with reduced growth in the number of nursing home beds \citep{harrington1997effect,swan1991certificate,zinn1994market}. We contribute to this literature by showing the causal effects of repealing nursing home CON, as well as its potential heterogeneous effects across states. Studies on the impact of nursing home CON laws on expenditure have found mixed results. \citet{grabowski2003effects} use a two-way fixed effects analysis and find that nursing home CON did not have a significant effect on Medicaid expenditures. Rahman et al \citet{rahman2016impact} exclude states that changed their CON regulations in their period of study and find that states with nursing home CON laws had faster Medicaid and Medicare expenditure growth than states without CON laws. \citet{bailey2019can} finds that the presence of any CON regulation (not necessarily nursing home CON) in a state is positively associated with total health expenditures, as well as total nursing home expenditure.\\
\indent The effects predicted by our model and supported by our empirical analysis are opposite to the ones found in the previous literature. We argue that the lack of a theoretical framework and different model specifications in previous studies are likely to have caused the different results. Bailey’s specification does not look at the impact of nursing home CON. Instead, it analyzes the impact of a dummy variable representing the presence of any CON regulation. Rahman et al exclude states that changed their CON regulations in their period of study and show the correlations between nursing home CON and expenditures, which does not have a causal interpretation. All previous analyses also don’t show the potential heterogeneous effects of changes in nursing home CON regulation, while our empirical strategy highlights the heterogeneous effects of this regulation. 

\end{document}