\documentclass[../Main.tex]{subfiles}

\begin{document}

CON laws were first introduced in 1964 by the state of New York and became widely popular among US states in the 1970s. Due to The National Health Planning and Resources Development Act of 1974 (P.L. 93–641), which heavily incentivized states to create CON regulations, all but one state (Louisiana) had some CON laws in place by 1980. This federal push for CON regulations was reversed in the early-1980s (P.L. 99–660, Title VII). Ten states repealed their CON laws soon after the reversal of this federal push, others (PA, IN, and ND) repealed their CON laws in the 1990s, and New Hampshire repealed its CON laws in 2016. All other states, as well as Louisiana (which adopted CON in 1991) and Indiana (which reinstated Nursing Home CON in 2019), currently have CON laws. This paper analyses the impact of the CON repeal in PA, IN, and ND because these are the only repeals that allow for sufficient pre-treatment and post-treatment time.\\
\indent Past research has analyzed the correlation between nursing home CON and expenditures, as well as its impact on the total quantity of nursing home beds. Multiple studies have found that the presence of CON laws reduced growth in the number of nursing home beds \citep{harrington1997effect,swan1991certificate,zinn1994market}. Studies on the impact of nursing home CON laws on expenditure have found mixed results. \citet{grabowski2003effects} use a two-way fixed effects analysis and find that nursing home CON did not have a significant effect on Medicaid expenditures. Rahman et al \citet{rahman2016impact} exclude states that changed their CON regulations in their period of study and find that states with nursing home CON laws had faster Medicaid and Medicare expenditure growth than states without CON laws. \citet{bailey2019can} finds that the presence of any CON regulation (not necessarily nursing home CON) in a state is positively associated with total health expenditures, as well as total nursing home expenditure.\\
\indent However, Bailey’s specification does not look at the impact of nursing home CON. Instead, it analyzes the impact of a dummy variable representing the presence of any CON regulation. Furthermore, both \citet{grabowski2003effects} and \citet{bailey2019can} apply a two-way fixed effect analysis with treatment at different times, which \citet{goodman2021difference} shows is biased if effects change over time, which is very likely the case. All previous analyses also don’t show the potential heterogeneous effects of changes in nursing home CON regulation on nursing home expenditure and nursing home quantity. We fill this gap by formulating a hypothesis of the potential heterogeneous impact of nursing home CON on expenditure and by supporting this hypothesis through a synthetic control analysis that estimates the effect of repealing nursing home CON in Pennsylvania, North Dakota, and Indiana.

\end{document}