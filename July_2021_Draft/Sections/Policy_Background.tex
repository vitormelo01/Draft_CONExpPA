\documentclass[../Main.tex]{subfiles}

\begin{document}


%\subsection{Background}
%CON laws were first introduced in 1964 by the state of New York and became widely popular among US states in the 1970s. Due to The National Health Planning and Resources Development Act of 1974 (P.L. 93–641), which heavily incentivized states to implement CON regulations, all but one state (Louisiana) had implemented CON laws in some of their health sectors by 1980. This federal push for CON regulations was reversed in the early-1980s (P.L. 99–660, Title VII). Ten states repealed their nursing home CON laws soon after the reversal of this federal push in the early 1980s, others (PA, IN, and ND) repealed their CON laws in the 1990s, and New Hampshire repealed its CON laws in 2016. All other states, as well as Louisiana (adopted nursing home CON in 1991), Connecticut (adopted nursing home CON in 1993) and Indiana (which reinstated Nursing Home CON in 2019), currently have nursing home CON laws. This paper analyses the impact of the CON repeal in PA, IN, and ND because these are the only repeals that allow for sufficient pre-treatment and post-treatment time, as well as potential control states.\footnote{Louisiana and Connecticut allow for sufficient pre-treatment and post-treatment time but there are no comparison units since these are the only states that did not have nursing home CON in the 1980s.}

%\subsection{Literature on Medicaid and Certificate-of-Need Laws}

%Previous research has explored different policies may affect Medicaid expenditure, and in particular Medicaid expenditure of nursing homes. \citet{grabowski2004recent} explore how the repeal of the Boren amendment, which gave states greater freedom to set Medicaid nursing home policy, affected Medicaid expenditures. \citet{goda2011impact} analyzes the effects of subsidies to private long-term care insurance on Medicaid expenditures. 

%Studies on the impact of nursing home CON laws on Medicaid expenditure have found mixed results. \citet{grabowski2003effects} use a two-way fixed effects analysis and find that nursing home CON did not have a significant effect on Medicaid expenditures. \citet{rahman2016impact} exclude states that changed their CON regulations in their period of study and find that states with nursing home CON laws had faster Medicaid and Medicare expenditure growth than states without CON laws. \citet{bailey2019can} finds that the presence of any CON regulation (not necessarily nursing home CON) in a state is positively associated with total health expenditures, as well as total nursing home expenditure. The effects predicted by our model and supported by our empirical analysis on total and Medicaid Expenditure are opposite to the ones found by \citet{bailey2019can}. However, Bailey’s specification does not look at the impact of nursing home CON. Instead, it analyzes the impact of the presence of any CON regulation.

%Many have analyzed the correlation between nursing home CON and expenditures, as well as quantity of nursing home services. Multiple studies have found that the presence of nursing home CON laws are associated with reduced growth in the number of nursing home beds \citep{harrington1997effect,swan1991certificate,zinn1994market}. We contribute to this literature by showing the causal effects of repealing nursing home CON, as well as its potential heterogeneous effects across states.   

%\subsection{Literature on Inter-Jurisdiction Competition}

%This paper is the first political economy analysis of nursing home CON laws and the first inter-jurisdictional competition model applied to the context of healthcare policy. However, much research has explored questions of competitive federalism. \citet{tiebout1956pure} was the first to show that competition across jurisdictions places competitive pressures on local governments that lead to predictable policy choices. He shows that, under certain conditions, competitive pressures lead governments to provide the optimal level of public goods. \citet{zodrow1986pigou} formalize the mechanisms of what became known as tax competition models. They model two countries competing for a perfectly mobile capital stock and show that, in equilibrium, tax rates are lower in both countries than they would be otherwise. \citet{bucovetsky1991asymmetric} and \citet{kanbur1993jeux} show that differences in country size can substantially change the predictions of the model such that smaller countries face stronger incentives to reduce taxes. \citet{genschel2002globalization} shows that institutional constraints may delay policy adjustment to competitive pressures. \citet{basinger2004remodeling} and \citet{plumper2009there} show that political costs mitigate the competitive downward pressure on tax rates. 

%A large literature also explores the presence of a race to the bottom in welfare benefits. \citet{gramlich1984migration} model the effect of migration on state welfare policy and find that competition across jurisdictions incentivizes political representatives to provide lower welfare benefits than they otherwise would. \citet{peterson1989american} explores the same question and develops the magnet hypothesis which proposes that an increase in state welfare benefits may lead to an increase in poverty rates because the state becomes a magnet that attracts the poor citizens from other states. \citet{saavedra2000model} develops a model analogous to the ones from the tax competition literature and shows that state choices of welfare benefits face a downward pressure from interstate competition. 

%We contribute to this literature by modeling how migration associated with the consumption of nursing home services leads to inter-state competition over CON restrictions. We adapt the structure of the tax competition model from \citet{basinger2004remodeling} to the context of nursing home CON regulation. Our model predicts that the repeal of nursing home CON will lead to an increase in the number of nursing homes services and Medicaid expenditures because states will become a "magnet" for nursing home consumers. This proposition is analogous to the welfare magnet hypothesis proposed by \citet{peterson1989american}. In the absence of political marginal costs, we show that equilibrium nursing home CON regulations spiral upward leading to a race to the top in health regulation. Once we consider political constraints, we find that political costs can mitigate this competitive upward pressure. Most importantly, the model shows that inter-state competition can incentivize state politicians to impose nursing home CON restrictions that are higher than they would be in the absence of such competition.




















\end{document}