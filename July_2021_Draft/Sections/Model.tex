\documentclass[../Main.tex]{subfiles}

\begin{document}

This section will model the competition of Medicaid eligible nursing homes for a consumers who qualifies for Medicaid assistance. We focus on Medicaid eligible nursing homes because around 90\% of nursing homes accept Medicaid patients. We focus on consumers who qualify for Medicaid because the overwhelming majority of nursing homes services are covered by Medicaid. In order to become eligible for Medicaid, a nursing home consumer must either have low income and assets or she must give up nearly all their income (outside of her home, car, personal belongings, or savings for funeral expenses) to the Medicaid system. If one decides to not give up her assets, she will have pay for the services she needs. However, less than 20\% of nursing home expenditures come from self-pay or private insurance, which implies that most consumers of nursing home services are able to qualify for Medicaid. 

\subsection{Preferences, Actions, and Technology of the Consumer}
Assume a representative nursing home consumer who qualifies for Medicaid. Thus, this consumer either has low income and assets or has already given up her assets to the Medici ad system in order to qualify. This consumer makes a discrete choice; that is, she chooses only one of the nursing homes available. Discrete models of consumption often do not assume a representative consumer. However, we follow other models of inter-jurisdiction mobility in assuming a representative agent because this assumption does not change any of the comparative statics and propositions presented in this section in addition to substantially simplifying the model. The particular set up in this model is analogous to one in \citet{basinger2004remodeling}.

The nursing homes are vertically differentiated based on quality where the consumer prefers higher quality nursing homes, all else equal. While nursing home quality is an important factor, it is certainly not the only factor. The benefits of choosing a given firm/state is likely to be determined by the weather, proximity to family, proximity to nature, among various other factors. We formalize this idea by adding a stochastic shock $v_i$, which represents the preferences of the consumer for a given state. This stochastic shock implies that the quality of nursing homes does not determine with certainty the choices of consumers. We assume that that all firms observe the biases and the probabilistic distribution of the stochastic shock, but do not observe its realization. Consumers have preferences $u(z_i,v_i) = v_i + z_i$ where $i$ indexes the firm/state. $z_i$ represents the quality of firm $i$ and $v_i$ represents the stochastic shock of firm/state $i$. Let $s_i$ represent the net benefit a nursing home consumer gets from choosing a given state $i$ such that $s_i \equiv v_i + z_i$. Since the consumer only chooses one nursing home, she chooses the nursing home with the highest $s_i$.

\subsection{Preferences, Actions, and Technology of the Firms}
Assume each state has a representative firm that offers nursing home services. The firms face a constant price $P$.
Recall that this model is focused only on Medicaid eligible consumers and nursing homes that accept Medicaid patients. If a consumer is eligible, Medicaid pays 100\% of the nursing home costs regardless of the consumer's health status. Nursing homes, unlike assisted living communities, do not line item their billings. The cost of care, room, meals, and medical supplies are all included in the daily rate. Medicaid pays a fixed daily rate so a nursing home Medicaid beneficiary does not have to pay any part of the cost. Thus, the price that nursing homes receive is effectively fixed and prices do not affect the decisions of consumers since Medicaid covers all nursing home costs.

Firms face a cost $C(x_i, z_i)$ such that $C'(x_i)=c_i$ where $c_i$ is constant and $C'(z_i)$ is increasing in $z_i$. Assume that for all $i$, $C'(z_i)>0$, $C''(z_i)>0$ and $C'(0)=0$. Thus, cost of quality increases continuously and at an accelerating rate. Each firm maximizes profits such that \begin{equation}\pi_i = P*x_i(v_i, z_i, v_{-i}, z_{-j}) -C(x_i, z_i) \end{equation} 
where $x_i$ represents the demand for a given nursing home $i$, which depends on the stochastic shock $v_i$ and the quality of nursing home $z_i$, as well as the quality of other nursing homes $z_{-i}$ and the stochastic shock of other states $v_{-i}$. The firm maximizes profits by choosing $z_i$ such that \begin{equation} \frac{\partial x_i}{\partial z_i^*}*(P-c_i)= C'(z_i) \end{equation} Assume that $P - c_i > 0$ such that each additional consumer that chooses firm $i$ brings in the net revenue $P - c_i$. Thus, firms compete for each consumer by choosing the level of quality where the probability of the consumer choosing firm $i$ is \begin{equation} x_{i}(z_i) \equiv p_i(z_i) \equiv p_i(z_i + v_i |z_{-i}, v_{-i}) \end{equation}
where that $p_{i}(z)$ is continuously differentiable, increasing in $z_i$ and $v_i$, decreasing in $z_{-i}$ and $v_{-i}$, and concave in $z_i$ such that for all $i$, $p_{i}'(z_i)\geq 0$ and $p_{i}''(z_i) \leq 0$.

\subsection{Timing, Information, and Equilibrium}
The set up here is a one time game such that players (firms) choose quality level $z_i$ simultaneously. After the firms make their choices, the stochastic shock $v_i$ is realized and then the consumer makes their choice. Assume that the utility function, payoffs, costs, strategies and types of all states are common knowledge. The appropriate equilibrium concept for this game is a Nash equilibrium. Thus, we assume that each state best responds with the knowledge that each opponent $j$ must choose its own optimal level of quality, $z_j^*$.

\subsection{Solving the Model Without CON}
The probability that a nursing home consumer chooses a state $i$ can be defined as \begin{equation}p_i(z) \equiv p(s_i> max(s_j), \forall j \neq i )\end{equation}
Thus, it must be that \begin{equation}p_i(z) \equiv \prod_{j\neq i} prob(v_i + z_i > v_j + z_j) \end{equation} 
Rearranging we get,
 \begin{equation}p_i(z) \equiv \prod_{j\neq i} prob(z_i-z_j > v_j - v_i) \end{equation} 
The bias between two states is denoted as $b_{ij} \equiv v_i -v_j $. $b_{ij}$ represents the bias in favor a state i and $b_{ji}$ represents the bias in favor of state j. Thus,\begin{equation}p_i(z) \equiv \prod_{j\neq i} prob(z_i -z_j > b_{ji}) \end{equation}  
Hence, a firm i solves the following:\begin{equation}max \rightarrow [\prod_{j\neq i} prob(z_i -z_j > b_{ji})(P-c_i)] - C(z_i)\end{equation}
Assuming an internal solution, we find 
 \begin{equation}\frac{\partial [\prod_{j\neq i} prob(z_i^* -z_j > b_{ji})]}{\partial z_i}(P-c_i) = C'(z_i) \end{equation}

\subsection{Incorporating CON Regulations into the Model}

We model CON regulations as an increase in the marginal cost of quantity such that firms with CON have $C'(x_i) = c_i + R$ where $c_i$ is the constant marginal cost of quantity  and $R$ is the additional cost of getting permission from government to offer an additional nursing home service. In that case, the firm in a state with NH-CON would maximize

\begin{equation} max \rightarrow [\prod_{j\neq i} prob(z_i -z_j > b_{ji})](P - c_i - R) - C(z_i) \end{equation}
Assuming an internal solution, we find that
 \begin{equation}\frac{\partial [\prod_{j\neq i} prob(z_i^* - z_j > b_{ji})]}{\partial z_i}(P - c_i - R) = C'(z_i) \end{equation}


%\subsection{Propositions}

%\subsubsection{Proposition 1:} 

\begin{Proposition}\label{prop1}
The implementation of CON in state $i$ will lead to a reduction in the equilibrium nursing home quality in $i$. Likewise, the repeal of CON in state $i$ will lead to an increase in the equilibrium nursing home quality in state $i$. 
\end{Proposition}


Recall that $p_{i}'(z_i)\geq 0$, $p_{i}''(z_i) \leq 0$, $c_z'(z_i)>0$, and $c_z''(z_i)>0$. Thus, the expected marginal benefit curve associated with $p_{i}'(z_i)$ is concave, while the marginal cost curve $c_z'(z_i)$ is assumed to be convex. Figure \ref{fig:model_graph} shows how an increase in marginal cost associated with NH-CON leads to a decrease in the marginal benefit of quality, which will lead to relatively lower equilibrium quality levels in CON states.


\begin{Proposition}\label{prop2}
The implementation of CON in state $i$ will lead to a reduction in the expected quantity of nursing homes services in state $i$. Likewise, the repeal of CON in state $i$ will lead to an increase in the expected quantity of nursing homes services in state $i$. 
\end{Proposition}

Proposition (\ref{prop1}) shows that the implementation of CON in state $i$ must decrease the quality level chosen by a given firm $i$. In an equilibrium with many Medicaid consumers, a reduction in quality will decrease the probability that each consumer comes to state $i$, which implies a lower expected quantity of nursing home services. The opposite would be true for a repeal of CON. 

\begin{Proposition} \label{prop3}
The implementation of CON in state $i$ will lead to a reduction Medicaid expenditures in state $i$. Likewise, the repeal of CON in state $i$ will lead to an increase in Medicaid expenditures in state $i$.  
\end{Proposition}

Recall that this model explores the competition of Medicaid eligible nursing homes for a consumers who qualifies for Medicaid assistance. If a consumer is eligible, Medicaid pays 100\% of the nursing home costs regardless of the consumer's health status. Thus, proposition (\ref{prop2}) implies that an increase in the quantity of nursing home services will lead to an increase in Medicaid expenditures.


\end{document}