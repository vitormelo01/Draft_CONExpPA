\documentclass[../Main.tex]{subfiles}

\begin{document}


This section models the competition of Medicaid eligible nursing homes for a consumers who qualifies for Medicaid assistance. We focus on Medicaid eligible nursing homes because around 90\% of nursing homes accept Medicaid patients. We focus on consumers who qualify for Medicaid because the overwhelming majority of nursing home services are covered by Medicaid. To become eligible for Medicaid, a nursing home consumer must either have low income and assets or she must give up nearly all their income (outside of her home, car, personal belongings, or savings for funeral expenses) to the Medicaid system. If one decides to not give up her assets, she will have pay for the services she needs. However, less than 20\% of nursing home expenditures come from self-pay or private insurance, which implies that most consumers of nursing home services qualify for Medicaid. 

\subsection{Preferences, Actions, and Technology of the Consumer}
Consider a dynamic game in which politicians from each state simultaneously choose their level of CON regulation; nursing homes from each state simultaneously choose their level of quality, after observing all states' regulators; consumers choose their nursing homes, after observing all nursing homes' quality. The appropriate solution concept for this game is subgame perfect Nash equilibrium (SPNE). We solve for the SPNE in standard fashion of analyzing players' choices in reverse temporal order. 

Assume a representative nursing home consumer who qualifies for Medicaid, meaning they either have low assets or have already given up their assets to the Medicaid system. The consumer chooses only one of the nursing homes available. Such discrete choice models of consumption often do not assume a representative consumer. However, we follow other models of inter-jurisdiction mobility in assuming a representative agent because this assumption does not change our conclusions and substantially simplifies the model. The particular set up in this model is analogous to \citet{basinger2004remodeling}.

The consumer's preferences over nursing homes depends on each nursing home's quality and factors such as weather, proximity to family, proximity to nature, among various others. We formalize these factors with a stochastic shock $s_i$, that represents the consumer's preferences for a given state $i$. This stochastic shock implies that the consumer's choice is not determined solely by the quality of the nursing home. Assume that firms do not observe the realization of shocks, but only its probability distribution. The consumer has preferences $u(s_i,z_i) = s_i + z_i$ where $i$ indexes the firm/state. $z_i$ represents the quality of firm $i$ and $s_i$ represents the stochastic shock of firm/state $i$. 

\subsection{Preferences, Actions, and Technology of the Firms}
Let each state have a representative firm that offers nursing home services. Assume N states where N is large. The firms face a constant price $P$. Recall that this model is focused on Medicaid eligible consumers and nursing homes that accept Medicaid patients. If a consumer is eligible, Medicaid pays 100\% of the nursing home costs regardless of the consumer's health status. Nursing homes, unlike assisted living communities, do not line item their billings. The cost of care, room, meals, and medical supplies are all included in the daily rate. Medicaid pays a fixed daily rate so a nursing home Medicaid beneficiary does not have to pay any part of the cost. Thus, the price that nursing homes receive is effectively fixed and prices do not affect the decisions of consumers since Medicaid covers all nursing home costs.

Firms face a cost $C_i(x_i, z_i)$ such that $C_i'(x_i)= c_i$ where $c_i$ is constant. Assume that $C_i'(z_i)$ is increasing in $z_i$ and that for all $i$, $C_i'(z_i)>0$, $C_i''(z_i)>0$ and $C_i'(0)=0$. Thus, cost of quality increases continuously and at an accelerating rate. Each firm maximizes profits such that \begin{equation}\pi_i = P*x_i(v_i, z_i, v_{-i}, z_{-j}) -C_i(x_i, z_i) \end{equation} 
where $x_i$ represents the demand for a given nursing home $i$, which depends on the stochastic shock $v_i$ and the quality of nursing home $z_i$, as well as the quality of other nursing homes $z_{-i}$ and the stochastic shock of other states $v_{-i}$. The firm maximizes profits by choosing $z_i$ such that \begin{equation} \frac{\partial x_i}{\partial z_i^*}*(P-c_i)= C_i'(z_i) \end{equation} Assume that $P - c_i > 0$ such that each additional consumer that chooses firm $i$ brings in the net revenue $P - c_i$. Thus, firms compete for each consumer by choosing the level of quality where the probability of the consumer choosing firm $i$ is \begin{equation} x_{i}(z_i) \equiv p_i(z_i) \equiv p_i(z_i + v_i |z_{-i}, v_{-i}) \end{equation}
where that $p_{i}(z)$ is continuously differentiable, increasing in $z_i$ and $v_i$, decreasing in $z_{-i}$ and $v_{-i}$, and concave in $z_i$ such that for all $i$, $p_{i}'(z_i)\geq 0$ and $p_{i}''(z_i) \leq 0$.

\subsection{Solving the Firm's Subgame Without CON}
The probability that a nursing home consumer chooses a state $i$ can be defined as \begin{equation}p_i(z) \equiv p(s_i + z_i > max(s_j + z_j), \forall j \neq i )\end{equation}
Thus, it must be that \begin{equation}p_i(z) \equiv \prod_{j\neq i} prob(s_i + z_i > s_j + z_j) \end{equation} 
Rearranging we get,
 \begin{equation}p_i(z) \equiv \prod_{j\neq i} prob(z_i-z_j > s_j - s_i) \end{equation} 
Let the bias between two states be denoted as $b_{ij} \equiv s_i - s_j $. $b_{ij}$ represents the bias in favor a state i and $b_{ji}$ represents the bias in favor of state j. Thus,\begin{equation}p_i(z) \equiv \prod_{j\neq i} prob(z_i -z_j > b_{ji}) \end{equation}  
Hence, a firm i solves the following:\begin{equation}max \rightarrow [\prod_{j\neq i} prob(z_i -z_j > b_{ji})(P-c_i)] - C(z_i)\end{equation}
Assuming an internal solution, we find 
 \begin{equation}\frac{\partial [\prod_{j\neq i} prob(z_i^* -z_j > b_{ji})]}{\partial z_i}(P-c_i) = C'(z_i) \end{equation}

\subsection{Solving the Firm's Subgame With CON}

We model CON regulations as an increase in the marginal cost of quantity such that firms with CON have $C'(x_i) = c_i + R$ where $c_i$ is the constant marginal cost of quantity  and $R$ is the additional cost of getting permission from government to offer an additional nursing home service. In that case, the firm in a state with nursing home CON would maximize

\begin{equation} max \rightarrow [\prod_{j\neq i} prob(z_i -z_j > b_{ji})](P - c_i - R) - C(z_i) \end{equation}
Assuming an internal solution, we find that
 \begin{equation}\frac{\partial [\prod_{j\neq i} prob(z_i^* - z_j > b_{ji})]}{\partial z_i}(P - c_i - R) = C'(z_i) \end{equation}


%\subsection{Propositions}

%\subsubsection{Proposition 1:} 

\begin{Proposition}\label{prop1}
The implementation of CON in state $i$ will lead to a reduction in the equilibrium nursing home quality in $i$. Likewise, the repeal of CON in state $i$ will lead to an increase in the equilibrium nursing home quality in state $i$. 
\end{Proposition}


Recall that $p_{i}'(z_i)\geq 0$, $p_{i}''(z_i) \leq 0$, $c_z'(z_i)>0$, and $c_z''(z_i)>0$. Thus, the expected marginal benefit curve associated with $p_{i}'(z_i)$ is concave, while the marginal cost curve $c_z'(z_i)$ is assumed to be convex. Figure \ref{fig:model_graph} shows how an increase in marginal cost associated with nursing home CON is associated decrease in the marginal benefit of quality. 

The effect an increase in CON on equilibrium quality of firm $i$ depends on both the effect on state $i$'s best response, as well how other states respond to the quality change in state $i$. The direct effect of CON on the best response function of firm $i$ is clear in Figure \ref{fig:model_graph}. An increase in CON leads state $i$ to decrease its quality which will weakly increase the probability of other firms receiving the representative consumer. Such increase in probability of other firms leads other states to increase their quality which leads to a further decrease in quality of firm $i$. Thus, an increase in CON will lead to lower equilibrium quality levels in the affected state. 


\begin{Proposition}\label{prop2}
The implementation of CON in state $i$ will lead to a reduction in the expected quantity of nursing homes services in state $i$. Likewise, the repeal of CON in state $i$ will lead to an increase in the expected quantity of nursing homes services in state $i$. 
\end{Proposition}

Proposition (\ref{prop1}) shows that the implementation of CON in state $i$ must decrease the quality level chosen by a given firm $i$. In an equilibrium with many Medicaid consumers, a reduction in quality will decrease the probability that each consumer comes to state $i$, which implies a lower expected quantity of nursing home services. The opposite would be true for a repeal of CON. 

\citet{peterson1989american} develop the magnet hypothesis in welfare interstate competition which proposes that, all else equal, an increase in states welfare policy will lead to an increase in poverty rates because the state becomes a magnet that attracts the poor citizens from other states. This hypothesis is analogous to proposition (\ref{prop2}). The repeal of nursing home CON in a given state $i$ will make the state a magnet for nursing home consumers from other states and for consumers who would otherwise leave state $i$.

\begin{Proposition} \label{prop3}
The implementation of CON in state $i$ will lead to a reduction in Medicaid expenditures in state $i$. Likewise, the repeal of CON in state $i$ will lead to an increase in Medicaid expenditures in state $i$.  
\end{Proposition}

Recall that this model explores the competition of Medicaid eligible nursing homes for a consumers who qualifies for Medicaid assistance. If a consumer is eligible, Medicaid pays 100\% of the nursing home costs regardless of the consumer's health status. Thus, the magnet hypothesis developed in proposition (\ref{prop2}) implies that an increase in the quantity of nursing home services will lead to an increase in Medicaid expenditures.

\subsection{Preferences, Actions, and Technology of Politicians}

The third group of players are representative politicians for each state. If the consumer chooses to go to state $i$, that state will have to pay for her Medicaid expenditure denoted as $M$. Let $f(M)$ denote the political gain of allocating sate expenditure to cover the Medicaid costs of a consumer such that $f(M)<0$. Thus, we assume there is a high opportunity cost to marginal Medicaid expenditure on nursing home residents since they are not an organized political group, they may not be from or have family in state $i$, and the associated cost per resident is very high. Suppose that N states compete to not receive the nursing home consumer by choosing a level of CON restrictions $R_i$ for state $i$. Let $R_i= \infty$ represent a moratorium where no additional beds are allowed in state $i$. 

States also face a cost to implementing CON restrictions $R_i$ such as the transactions cost of passing legislation, the political cost of making it harder for sick people to find nursing homes in their states, among others. Let $c_i(R_i)$ represent the cost of its CON restrictions $R_i$. Assume that for all $i$, $c_i'(R_i)>0$, $c_i''(R_i)>0$ and $c_i(0)=0$. Thus, costs increase continuously and at an accelerating rate.

\subsection{Solving the Politicians' Subgame}

Given the preferences and actions of the representative politician, a state i solves the following:\begin{equation}max \rightarrow p_i(R_i|R_{-i}^*)f(M) -c_i(R_i)\end{equation} where $p_i(R_i|R_{-i}^*)$ is strictly positive, as well as decreasing and convex in $R_i$. Assuming an interior solution, we can evaluate the first order conditions by taking the first derivative with respect to $R_i$ and setting it equal to 0 to find \begin{equation}p_i'(R_i^*|R_{-i}^*)f(M) = c_i'(R_i^*)\end{equation} 

 \begin{Proposition}
In the absence of political marginal costs, equilibrium nursing home CON regulations will spiral upward leading to a race to the top in health regulation such that all states have a moratorium of nursing home beds. 
\end{Proposition}

Recall that $p_i(R_i|R_{-i}^*)$ is strictly decreasing in $R_i$. In the absence of marginal political costs associated with this regulation, states costlessly increase $R_i$ in an effort to diverge some nursing home consumers to other states. Since other states behave in the same way, the only possible Nash equilibrium is where $R_i^* = R_j^* = \infty $ for all states. Nursing home CON regulations where $R_j^* = \infty $ are equivalent to the moratorium policy, where no new nursing home beds are allowed to be offered.\footnote{Some states such as Connecticut and Wisconsin currently have a moratorium in nursing homes beds, while many others either have or have had the same level of restriction. Thus, this proposition under the assumption of no political costs of CON is not as extreme as it may seem.} This result is analogous to the race to the bottom results found in the canonical tax competition models \citep{zodrow1986pigou, basinger2004remodeling} and the welfare competition models \citep{gramlich1984migration,peterson1989american,schram1998without}. This proposition implies that in the absence of political costs, all states would have a moratoria policies where no new nursing home beds are allowed.


\begin{Proposition}
Political costs mitigate the upward pressure to increase nursing home regulation such that $R_i$ may be finite. 
\end{Proposition}

Once political costs are incorporated in the model, states will choose a finite level of nursing home CON. Recall that $c_i'(R_i^*)$ is assumed to be strictly increasing and convex. Thus, this result model implies that states might not all apply moratoria. The equilibrium level of nursing home CON will depend on the magnitude of $p_i'(R_i^*|R_{-i}^*)$ and $c_i'(R_i^*)$. However, the model shows that interstate competition to avoid nursing home Medicaid costs creates incentives for all states to impose nursing home CON restrictions higher than they would impose in the absence of such competition.







\end{document}