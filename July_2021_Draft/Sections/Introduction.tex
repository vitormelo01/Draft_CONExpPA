\documentclass[../Main.tex]{subfiles}

\begin{document}

Nursing home services have historically represented over a third of Medicaid expenditures \citep{wiener1999controlling}. Medicaid is the primary payer for nursing homes, covering over 60 percent of all nursing home residents \defcitealias{ahca2020pressrelease}{AHCA/NCAL,~2020}\citepalias{ahca2020pressrelease}. As a result, various states attempted to reduce nursing home Medicaid expenditures by restricting the number of nursing homes and nursing home beds in their respective states through Certificate-of-need (CON) laws \citep{feder1980regulating,grabowski2003effects,rahman2016impact}. CON laws impose bureaucratic costs to healthcare providers in 35 American states by requiring healthcare providers planning on opening a new healthcare facility, expanding a current facility, or purchasing medical equipment to first show to a regulatory body that their region needs the service.\footnote{In this paper, the term nursing home (often called skilled nursing homes or long-term care facilities) refers only to in-patient rehabilitation and medical treatment centers staffed with trained medical professionals.}\\
\indent Nursing home CON laws, however, have been both the most restrictive and most common among states with any CON regulations. While all CON regulations impose a bureaucratic cost of applying for a license, it is still often possible to get the license and offer the new service for most categories of CON regulations.  Nursing home CON, however, works in many states as a moratorium, which implies that the provision of any new nursing home services is prohibited. Additionally, even states that do not have a moratorium tend to be much more restrictive with nursing home CON than other categories of CON \citep{american2020american}.\\
\indent Furthermore, nursing home CON regulations have historically been the first CON regulation to be adopted and the last to revoked. This means that virtually every state that had any CON laws, included nursing homes as one of its regulated sectors. Ohio, for example, dropped all its CON regulations (covering a total of 10 sectors) in 1997 except for its CON regulation on nursing homes. Indiana revoked all its CON regulations in 1999 but reinstated only its nursing home CON in 2019. Emmett Reed, Florida Health Care Association’s executive director, argued that the nursing home CON repeal in Indiana led to overproduction in the development of new nursing facilities which put a strain on the state’s Medicaid budget, ultimately leading lawmakers to backtrack and reinstate its CON law in 2018 (Add citation). He further concluded that nursing home CON must be maintained because “the burden [of repealing CON] ultimately will fall to the taxpayers” \citep{sexton2019conrepeal}. Therefore, regulators tend to view nursing home CON as one of, if not the most, necessary CON restriction.\footnote{For a complete picture of which states did and did not have NH-CON regulations during our sample period, see Figure \ref{fig:nh_con_map}.}\\
\indent This paper develops a model where firms compete for a mobile nursing home consumer and shows that NH-CON may push some of states’ Medicaid expenditure towards other states. We argue that what makes nursing home CON unique among CON regulations is that nursing home consumers often substitute their consumption toward neighboring states. Many nursing home consumers bring their residency to the new state, along with the Medicaid expenditure associated with their nursing home services.\footnote{Nursing Home consumers on Medicaid are required to change their residency to receive Medicaid benefits in other states. Since Medicaid covers the expenditure of most nursing home patients, most nursing home migration will be associated with the new state paying for the medicaid expenditures that the other state previously paid.} This is likely not the case with health clinics, hospitals, and urgent care centers. Hence, nursing home CON can be used by states to  push some of their citizens, along with their Medicaid expenditures, to other states.\\
\indent We support the predictions of this model by analyzing the effect of removing NH-CON regulations on the quantity of nursing homes and nursing home beds, as well as total expenditure and Medicaid expenditure. We apply a synthetic control analysis of the repeal of nursing home CON in Pennsylvania (PA), Indiana (IN), and North Dakota (ND). We find that removing nursing home CON regulations causes an increase in the quantity of nursing homes and nursing home beds in all three states. We also find that the repeal of nursing home CON led to a substantial and significant increase in total government nursing home expenditure and Medicaid nursing home expenditure in Pennsylvania, but results are insignificant in Indiana or North Dakota. These results are consistent with the predictions of our model and consistent with the idea that the current Medicaid system is set up in a way that nursing home CON can be used by states to push some of their Medicaid expenditure to other states. \\
\indent The paper is organized as follows: section two overviews the policy background of CON laws and overviews its literature with focus on nursing homes; section three describes the model used to understand the impact of nursing home CON on Medicaid expenditures; section four describes the data sources used; section five summarizes the empirical model utilized in this paper; section six highlights the results of the analysis; and section seven offers a conclusion and the policy implications of this research.


\end{document}