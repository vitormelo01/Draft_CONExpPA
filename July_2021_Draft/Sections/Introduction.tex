\documentclass[../Main.tex]{subfiles}

\begin{document}

Nursing home services have historically represented over a third of Medicaid expenditures \citep{wiener1999controlling}. Medicaid is the primary payer for nursing homes, covering about two thirds of all nursing home residents \defcitealias{ahca2020pressrelease}{AHCA/NCAL,~2020}\citepalias{ahca2020pressrelease}. Nursing homes are often regulated via certificate-of-need (CON) laws. CON laws impose bureaucratic costs to healthcare providers in 35 American states by requiring healthcare providers planning on opening a new healthcare facility, expanding a current facility, or purchasing medical equipment to first show to a regulatory body that their region needs the service. Nursing home Certificate-of-need (NH-CON) laws, however, have been both the most restrictive and most common among states with any CON regulations.\footnote{In this paper, the term nursing home (often called skilled nursing homes or long-term care facilities) refers only to in-patient rehabilitation and medical treatment centers staffed with trained medical professionals. Restriction levels in this paper are defined as the additional cost that firms have to incur to provide a nursing home service. These are often associated with high fees, ``red-tape,'' and a low probability of receiving authorization to offer the service. Thus, an increase in restrictions implies that the cost of applying for a CON went up or that the probability of receiving the CON went down.} While all CON regulations impose a bureaucratic cost of applying for a license, it is still often possible to get the license and offer the new service for most categories of CON regulations.  NH-CON, however, works in many states as a moratorium, which implies that the provision of any new nursing home services is prohibited. Even states that do not have a moratorium still require nursing homes to go through a long and costly review process to get a permit for any additional beds. \citep{american2020american}.\\
\indent Furthermore, nursing home CON laws have historically been the first CON regulation to be adopted and the last to be repealed. Virtually every state that had any CON laws included nursing homes as one of its regulated sectors. Ohio, for example, dropped all its CON regulations (covering a total of 10 sectors) in 1997 except for its CON regulation on nursing homes. Indiana revoked all its CON regulations in 1999 but reinstated only its NH-CON in 2019. Emmett Reed, Florida Health Care Association’s executive director, argued that the NH-CON repeal in Indiana led to overproduction in the development of new nursing facilities which put a strain on the state’s Medicaid budget, ultimately leading lawmakers to backtrack and reinstate its CON law in 2019. He further concluded that NH-CON must be maintained because “the burden [of repealing CON] ultimately will fall to the taxpayers” \citep{sexton2019conrepeal}. Therefore, regulators tend to view NH-CON as one of, if not the most, necessary CON restriction.\footnote{For a complete picture of which states did and did not have NH-CON regulations during our sample period, see Figure \ref{fig:nh_con_map}.}\\
\indent This paper develops a model where firms compete for a mobile nursing home consumer and shows that NH-CON may push some of states’ Medicaid expenditure towards other states, and that repealing this regulation will lead to an increase in the quantity of nursing home services, as well as an increase in total and Medicaid expenditures on nursing home services. We argue that nursing home are unique among health services because individuals in need of nursing home services often move across states to get these services. This is especially true for consumers who live near state borders or who have family members who live in other states. Consumers of other health services may travel across states to get these services but they rarely move because of it. This distinction is essential because Medicaid expenditures are tied to one's residency. Therefore, when a nursing home consumer on Medicaid moves to a new state, she will have to update her residency and the new state will pay for her Medicaid costs.\\
\indent We support the predictions of this model by analyzing the effect of removing NH-CON regulations on the quantity of nursing homes and nursing home beds, as well as total nursing home expenditure and Medicaid nursing home expenditure. We apply a synthetic difference in differences analysis of the repeal of nursing home CON in Pennsylvania (PA), Indiana (IN), and North Dakota (ND). We find that removing nursing home CON regulations causes an increase in the quantity of nursing homes and nursing home beds, particularly in PA and IN. We also find that the repeal of NH-CON led to a significant increase in total government nursing home expenditure in PA and an increase in Medicaid nursing home expenditure in PA and IN. All significant results where we find evidence for pre-treatment parallel trends are consistent with the predictions of our model and consistent with the idea that the current Medicaid system is set up in a way that NH-CON can be used by states to push some of their Medicaid expenditure to other states. \\
\indent The paper is organized as follows: Section \ref{policy_background} overviews the policy background of CON laws and its literature with a focus on nursing homes; Section \ref{model} describes the model used to understand the impact of  NH-CON on access to nursing home services and nursing home expenditures; Section \ref{data} describes the data sources used; Section \ref{empirical_strategy} summarizes the synthetic difference in differences method and contrasts this empirical strategy with the traditional synthetic control and difference in differences methods; Section \ref{results} highlights the results of the synthetic difference-in-difference analysis and also contrasts it with the results from the synthetic control and difference in differences analyses; Section \ref{conclusion} offers a conclusion and the policy implications of this research.


\end{document}