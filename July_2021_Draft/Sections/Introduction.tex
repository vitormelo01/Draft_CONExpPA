\documentclass[../Main.tex]{subfiles}

\begin{document}

Nursing home services have historically represented over a third of Medicaid expenditures \citep{wiener1999controlling}. Medicaid is the primary payer for nursing homes, covering over 60 percent of all nursing home residents \defcitealias{ahca2020pressrelease}{AHCA/NCAL,~2020}\citepalias{ahca2020pressrelease}. As a result, various states attempted to reduce nursing home Medicaid expenditures by restricting the number of nursing homes and nursing home beds in their respective states through Certificate-of-need (CON) laws \citep{feder1980regulating,grabowski2003effects,rahman2016impact}. CON laws impose bureaucratic costs to healthcare providers in 35 American states by requiring healthcare providers planning on opening a new healthcare facility, expanding a current facility, or purchasing medical equipment to first show to a regulatory body that their region needs the service.\footnote{In this paper, the term nursing home (often called skilled nursing homes or long-term care facilities) refers only to in-patient rehabilitation and medical treatment centers staffed with trained medical professionals.}\\
\indent Nursing home CON laws, however, have been both the most restrictive and most common among states with any CON regulations. While all CON regulations impose a bureaucratic cost of applying for a license, it is still often possible to get the license and offer the new service for most categories of CON regulations.  Nursing home CON, however, works in many states as a moratorium, which implies that the provision of any new nursing home services is prohibited. Additionally, even states that do not have a moratorium tend to be much more restrictive with nursing home CON than other categories of CON \citep{american2020american}.\\
\indent Furthermore, nursing home CON regulations have historically been the first CON regulation to be adopted and the last to revoked. This means that virtually every state that had any CON laws, included nursing homes as one of its regulated sectors. Ohio, for example, dropped all its CON regulations (covering a total of 10 sectors) in 1997 except for its CON regulation on nursing homes. Indiana revoked all its CON regulations in 1999 but reinstated only its nursing home CON in 2019. Emmett Reed, Florida Health Care Association’s executive director, argued that the nursing home CON repeal in Indiana led to overproduction in the development of new nursing facilities which put a strain on the state’s Medicaid budget, ultimately leading lawmakers to backtrack and reinstate its CON law in 2018 (Add citation). He further concluded that nursing home CON must be maintained because “the burden [of repealing CON] ultimately will fall to the taxpayers” (Sexton, 2019 - add this citation to the bibtex file). Therefore, regulators tend to view nursing home CON as one of, if not the most, necessary CON restriction.\\
\indent We argue that what makes nursing home CON unique among CON regulations is that individuals in need of nursing home services can easily utilize these services in neighboring states. When choosing a nursing home for a family member, it is often not very costly to choose a nursing home a few hours away. This is likely not the case with health clinics, hospitals, and urgent care centers since most people would not consider driving a few hours for a doctor’s appointment or for emergency care. Hence, nursing homes in neighboring states are often a close substitute to nursing homes in one’s home state. Nursing home CON can artificially restrict the supply of nursing homes, which will lead potential nursing home consumers from one’s state to substitute towards nursing homes in neighboring states. Consequently, states can push some of their citizens, along with their Medicaid expenditures, to other states.\\
\indent But not all states provide close substitutes to a large group of people in neighboring states. We will categorize some states as “bad nursing home substitutes” and others as “good nursing home substitutes”. Bad nursing home substitutes tend to be states with low population density, low urban population, relatively cold weather, and with no major city near a large metropolis. Good nursing home substitutes tend be near large urban centers, similar/warmer weather to its neighbors, and have a major city near a large metropolis. Therefore, the impact of nursing home CON in each state depends on how good of a substitute its nursing homes are to nursing homes in neighboring states. Our hypothesis is that because potential nursing home consumers can migrate across states, repealing nursing home CON will lead to a large increase in total and Medicaid nursing home expenditures, as well as the quantity of nursing homes and nursing home beds, in states that can provide good nursing home substitutes. However, repealing nursing home CON will have a relatively smaller impact in states that can only provide bad nursing home substitutes. In other words, states neighboring a good nursing home substitute state without nursing home CON will be able to push some of their potential nursing home consumers, along with their Medicaid expenditures, to its neighbor state. States neighboring only good nursing home substitute states with nursing home CON and/or bad nursing home substitute states will not be able to push some of their Medicaid expenditures to its neighbor state.  \\
\indent We examine this hypothesis by applying a synthetic control analysis of the repeal of nursing home CON in Pennsylvania (PA), Indiana (IN), and North Dakota (ND). Since PA has two major cities, one of which (Philadelphia) is close to several large metropolitan centers (i.e. New jersey, New York City, Baltimore, Washington DC), it can likely offer a good substitute to potential nursing home consumers in CON neighboring states. IN and ND are less likely to offer a good substitute. We find that the repeal of Nursing Home CON in Pennsylvania led to a substantial increase in total nursing home expenditure, Medicaid nursing home expenditure, quantity of nursing homes, and quantity of nursing home beds. Our placebo test shows that all those impacts are statistically significant. This result is consistent with the hypothesis that potential nursing home consumers in states neighboring PA are substituting towards nursing homes in PA, which implies that PA is paying some Medicaid costs that in the absence of CON would be paid by its neighbors.\\
\indent We find that the repeal of CON in IN and ND led to an increase in the quantity of nursing homes and quantity of nursing home beds, but we find no impact on total nursing home expenditure and Medicaid nursing home expenditure. These results are consistent with our hypothesis since IN and ND are relatively less likely to offer close nursing home substitutes to others, which means fewer potential nursing home consumers migrated these states. We contribute to the CON literature by providing the first analysis of the potential heterogeneous effects of nursing home CON on expenditures and by showing that PA may effectively be paying for some of the Medicaid expenditure of its neighboring states.\\
\indent The paper is organized as follows: section two overviews the policy background of CON laws and overviews its literature with focus on nursing homes; section three describes the data sources used; section four summarizes the empirical model utilized in this paper; section five highlights the results of the analysis; and section six offers a conclusion.


\end{document}