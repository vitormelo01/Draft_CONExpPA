\documentclass[../Main.tex]{subfiles}

\begin{document}

Certificate of Need (CON) laws impose bureaucratic costs by requiring healthcare providers who plan on opening a new healthcare facility, expanding a current facility, or purchasing medical equipment to first show to a regulatory body that their region needs the service. Nursing home CON have historically been the most prevalent CON regulation. Figure \ref{fig:nh_histogram} shows that 36 American states regulate nursing homes via CON laws, which is more than any other health service. Often, CON regulations on nursing homes are also more restrictive than on all other health services. While all CON regulations impose a bureaucratic cost of applying for a license, it is still possible to get the license and offer the new service for most categories of CON regulations. Nursing home CON, however, has been applied in many states as a moratorium, which means that the provision of any new nursing home services is prohibited. Even states that do not have a moratorium still require nursing homes to go through a long and costly review process to get a permit for any additional beds \citep{american2020american}. Nursing home CON laws have also historically been the first CON regulation to be adopted and the last to be repealed. Ohio, for example, dropped all its CON regulations in 1997 except for its CON regulation on nursing homes. Indiana revoked all its CON regulations in 1999 but reinstated only nursing home CON in 2019. Thus, nursing homes are heavily regulated by CON laws to an extent that no other health service is. 
 
  %\footnote{In this paper, the term nursing home (often called skilled nursing homes or long-term care facilities) refers only to in-patient rehabilitation and medical treatment centers staffed with trained medical professionals. Restriction levels in this paper are defined as the additional cost that firms have to incur to provide a nursing home service. These are often associated with high fees, ``red-tape,'' and a low probability of receiving authorization to offer the service. Thus, an increase in restrictions implies that the cost of applying for a CON went up or that the probability of receiving the CON went down.} 

Nursing home services have historically represented over a third of Medicaid expenditures \citep{wiener1999controlling}. Medicaid is also the primary payer for nursing homes, covering about two thirds of all nursing home residents \defcitealias{ahca2020pressrelease}{AHCA/NCAL,~2020}\citepalias{ahca2020pressrelease}. The connection between nursing homes and Medicaid is often cited by politicians and healthcare executives as the primary reason to regulate nursing homes via CON. Emmett Reed, Florida Health Care Association’s executive director, argued that the repeal of nursing home CON in Indiana led to overproduction in the development of new nursing facilities which put a strain on the state’s Medicaid budget, ultimately leading lawmakers to backtrack and reinstate its CON law in 2019. He further concluded that nursing home CON must be maintained because “the burden [of repealing CON] ultimately will fall to the taxpayers” \citep{sexton2019conrepeal}. Thus, many regulators tend to view nursing home CON as one of, if not the most, necessary CON restriction largely due its presumed effect on Medicaid expenditures. 

This paper estimates the causal effects of nursing home CON and explores why this regulation is the most prevalent and often the most restrictive CON regulation. We develop a dynamic model in which politicians choose their level of CON regulation; nursing homes choose their level of quality, after observing all states' regulators; and consumers choose their nursing homes, after observing all nursing homes' quality. We find that repealing this regulation leads to an increase in nursing home quality, quantity, and Medicaid expenditures on nursing home services. The model also shows that interstate competition to avoid nursing home Medicaid costs creates incentives for state politicians to impose nursing home CON restrictions higher than they would impose in the absence of such competition. 

Our predictions and theoretical framework suggest that the decision of state politicians to regulate nursing homes through CON are analogous to decisions over taxing capital investments and offering welfare benefits. If a state taxes capital investments at lower rates than its neighbors, more capital investments will be made in this state, all else equal \citep{zodrow1986pigou,bucovetsky1991asymmetric, kanbur1993jeux, basinger2004remodeling, plumper2009there}. If a state offers less welfare transfers than other states, more people who would benefit from such transfers will move to that state, all else equal \citep{gramlich1984migration, peterson1989american, saavedra2000model}. Likewise, we find that if a state imposes lower nursing home CON restrictions than other states, more nursing home consumers will move to that state, all else equal. Therefore, repealing or reducing nursing home CON restrictions is expected to cause an increase in Medicaid expenditures.

All predictions from our model are specific to the nursing home market. Nursing homes are unique among health services because individuals in need of nursing home services often move across states to get these services. This is especially true for consumers who live near state borders or who have family members who live in other states. Consumers of other health services may travel across states to get these services but rarely move because of it. This distinction is essential because Medicaid expenditures are tied to one's residency. When a nursing home consumer who is eligible for Medicaid moves to a new state, she has to update her residency and the new state pays for her Medicaid expenses.  

We support the predictions from our model by analyzing the effect of removing nursing home CON regulations on the quantity of nursing homes, nursing home beds, specialized nursing home beds, Medicaid nursing home expenditure, and total nursing home expenditure. We apply a synthetic difference in differences analysis of the repeal of nursing home CON in Pennsylvania, Indiana, and North Dakota. Figure \ref{fig:nh_con_map} shows a complete picture of which states did and did not have nursing home CON regulations during our sample period. We find that the repeal of nursing home CON laws in Pennsylvania led to an increase in nursing home Medicaid expenditures of \$1.32 billion which represents a 70\% increase relative to its pre-treatment mean expenditure. The repeal of nursing CON laws in Indiana led to an increase in nursing home Medicaid expenditures of \$334 million which represents a 40\% increase relative to its pre-treatment mean nursing home Medicaid expenditure. The effect of the repeal in North Dakota, however, is unreliable due to a clear lack of pre-treatment parallel trends. 

We find that repealing nursing CON caused an increase in the number of nursing home beds which is consistent with the predictions of our model. The repeal also led to a substantial increase in the development of new nursing homes, implying that facilities in treated states soon after the repeal were much newer than in their counterfactual, which factors into consumers' perception of nursing home quality. Repealing CON also caused an increase in specialized care beds, which \citet{grabowski2010quality} show is correlated with higher quality of service. 

We also find that the effect of repealing this regulation on total nursing home expenditures is nearly the same as the effect on nursing home Medicaid expenditures which suggests that most, if not all, of the effect of repealing CON on expenditure is coming from its effect on Medicaid. We further analyze this hypothesis by testing if the repeal had any effect on Medicare and find that no evidence that it did. All results where we find evidence for parallel trends are consistent with the predictions of our model. These empirical results show that the effect of repealing nursing home CON is substantial and economically meaningful. Our model and empirical evidence are consistent with the proposition that inter-state competition within the current Medicaid system incentivizes state politicians to impose nursing home CON restrictions that are higher than they would be in the absence of inter-state competition.

We contribute to a literature that explores how different policies affect Medicaid expenditure, and in particular Medicaid expenditure on nursing homes. \citet{grabowski2004recent} show how the repeal of the Boren amendment, which gave states greater freedom to set Medicaid nursing home policies, affected Medicaid expenditures. \citet{goda2011impact} analyzes the effects of subsidies to private long-term care insurance on Medicaid expenditures. \citet{grabowski2003effects} use a two-way fixed effects analysis and find that nursing home CON did not have a significant effect on Medicaid expenditures. \citet{rahman2016impact} exclude states that changed their CON regulations in their period of study and find that states with nursing home CON laws had faster Medicaid and Medicare expenditure growth than states without CON laws. \citet{bailey2019can} finds that the presence of any CON regulation (not necessarily nursing home CON) in a state is positively associated with total health expenditures and total nursing home expenditure, which is opposite to the effects predicted by our model and supported by our empirical analysis. However, Bailey’s specification analyzes the impact of the presence of any CON regulation as opposed to nursing home CON. Many have also analyzed the correlation between nursing home CON and expenditures, as well as quantity of nursing home services. \citet{harrington1997effect,swan1991certificate,zinn1994market} find that the presence of nursing home CON laws are associated with reduced growth in the number of nursing home beds. We contribute to this literature by showing the causal effects of repealing nursing home CON on Medicaid Expenditure and quantity of nursing home beds, as well as its potential heterogeneous effects across states.   

We also offer the first political economy analysis of nursing home CON laws and the first inter-jurisdictional competition model applied to the context of healthcare policy. Yet, much research has explored questions of competitive federalism. \citet{tiebout1956pure} was the first to show that competition across jurisdictions places competitive pressures on local governments that lead to predictable policy choices. He shows that, under certain conditions, competitive pressures lead governments to provide the optimal level of public goods. \citet{zodrow1986pigou} formalize the mechanisms of what became known as tax competition models. They model two countries competing for a perfectly mobile capital stock and show that, in equilibrium, tax rates are lower in both countries than they would be otherwise. \citet{bucovetsky1991asymmetric} and \citet{kanbur1993jeux} show that differences in country size can substantially change the predictions of the model such that smaller countries face stronger incentives to reduce taxes. \citet{genschel2002globalization} explore how institutional constraints affect policy adjustment to competitive pressures. \citet{basinger2004remodeling} and \citet{plumper2009there} show that political costs mitigate the competitive downward pressure on tax rates. 

A large literature also explores the effects of inter-state competition on political decisions over welfare policies. \citet{gramlich1984migration} model the effect of migration on state welfare policy and find that competition across jurisdictions incentivizes political representatives to provide lower welfare benefits than they otherwise would. \citet{peterson1989american} explores the same question and develops the magnet hypothesis which proposes that an increase in state welfare benefits may lead to an increase in poverty rates because the state becomes a magnet that attracts some low-income citizens from other states. Our model predicts that the repeal of nursing home CON will lead to an increase in the number of nursing homes services and Medicaid expenditures because states will become a ``magnet" for nursing home consumers. This proposition is analogous to the welfare magnet hypothesis proposed by \citet{peterson1989american}. \citet{saavedra2000model} develops a model analogous to the ones from the tax competition literature and shows that state choices of welfare benefits face a downward pressure from interstate competition. We contribute to the literature on competitive federalism by modeling how migration associated with the consumption of nursing home services leads to inter-state competition over nursing home CON restrictions. 

%We adapt the structure of the tax competition model from \citet{basinger2004remodeling} to the context of nursing home CON regulation. Our model predicts that the repeal of nursing home CON will lead to an increase in the number of nursing homes services and Medicaid expenditures because states will become a ``magnet" for nursing home consumers. This proposition is analogous to the welfare magnet hypothesis proposed by \citet{peterson1989american}. In the absence of political marginal costs, we show that equilibrium nursing home CON regulations spiral upward leading to a race to the top in health regulation. Once we consider political constraints, we find that political costs can mitigate this competitive upward pressure. Most importantly, the model shows that inter-state competition can incentivize state politicians to impose nursing home CON restrictions that are higher than they would be in the absence of such competition.

%The paper is organized as follows: Section \ref{policy_background} overviews the policy background of CON laws and its literature with a focus on nursing homes; Section \ref{model} describes the model used to understand the impact of nursing home CON on access to nursing home services and nursing home expenditures; Section \ref{data} describes the data sources used; Section \ref{empirical_strategy} summarizes the synthetic difference in differences method and contrasts this empirical strategy with the traditional synthetic control and difference in differences methods; Section \ref{results} highlights the results of the synthetic difference-in-difference analysis and also contrasts it with the results from the synthetic control and difference in differences analyses; Section \ref{conclusion} offers a conclusion and the policy implications of this research.


\end{document}