\documentclass[../Main.tex]{subfiles}

\begin{document}

 Certificate of Need (CON) laws impose bureaucratic costs by requiring healthcare providers who plan on opening a new healthcare facility, expanding a current facility, or purchasing medical equipment to first show to a regulatory body that their region needs the service. Nursing home CON laws have historically been the most prevalent CON regulation. Figure \ref{fig:nh_histogram} shows that 36 American states regulate nursing homes via CON laws, which is more than any other health service. Often, CON regulations on nursing homes are also more restrictive than on all other health services. While all CON regulations impose a bureaucratic cost of applying for a license, it is still often possible to get the license and offer the new service for most categories of CON regulations. Nursing home CON, however, has be applied in many states as a moratorium, which implies that the provision of any new nursing home services is prohibited. Even states that do not have a moratorium still require nursing homes to go through a long and costly review process to get a permit for any additional beds \citep{american2020american}. 
 
 Nursing home CON laws have also historically been the first CON regulation to be adopted and the last to be repealed. Ohio, for example, dropped all its CON regulations (covering a total of 10 sectors) in 1997 except for its CON regulation on nursing homes. Indiana revoked all its CON regulations in 1999 but reinstated only nursing home CON in 2019. Thus, nursing homes are heavily regulated by CON laws to an extent that no other health service is. What is so different about the nursing home market that explains why its CON regulations are more prevalent and frequently more restrictive than other health sectors? Nursing homes are unique among health services because individuals in need of nursing home services often move across states to get these services. This is especially true for consumers who live near state borders or who have family members who live in other states. Consumers of other health services may travel across states to get these services but rarely move because of it. This distinction is essential because Medicaid expenditures are tied to one's residency. When a nursing home consumer who is eligible for Medicaid moves to a new state, she has to update her residency and the new state pays for her Medicaid expenses.  
 
  %\footnote{In this paper, the term nursing home (often called skilled nursing homes or long-term care facilities) refers only to in-patient rehabilitation and medical treatment centers staffed with trained medical professionals. Restriction levels in this paper are defined as the additional cost that firms have to incur to provide a nursing home service. These are often associated with high fees, ``red-tape,'' and a low probability of receiving authorization to offer the service. Thus, an increase in restrictions implies that the cost of applying for a CON went up or that the probability of receiving the CON went down.} 

Nursing home services have historically represented over a third of Medicaid expenditures \citep{wiener1999controlling}. Medicaid is also the primary payer for nursing homes, covering about two thirds of all nursing home residents \defcitealias{ahca2020pressrelease}{AHCA/NCAL,~2020}\citepalias{ahca2020pressrelease}. The connection between nursing homes and Medicaid is often cited by politicians and healthcare executives as the primary reason to regulate nursing homes via CON. Emmett Reed, Florida Health Care Association’s executive director, argued that the repeal of nursing home CON in Indiana led to overproduction in the development of new nursing facilities which put a strain on the state’s Medicaid budget, ultimately leading lawmakers to backtrack and reinstate its CON law in 2019. He further concluded that NH-CON must be maintained because “the burden [of repealing CON] ultimately will fall to the taxpayers” \citep{sexton2019conrepeal}. Thus, many regulators tend to view nursing home CON as one of, if not the most, necessary CON restriction largely due its presumed effect on Medicaid expenditures.

This paper develops a model where firms compete for a mobile nursing home consumer and shows that repealing this regulation will lead to an increase in the quantity of nursing home services, as well as an increase in total and Medicaid expenditures on nursing home services. We support the predictions of this model by analyzing the effect of removing NH-CON regulations on the quantity of nursing homes and nursing home beds, as well as total nursing home expenditure and Medicaid nursing home expenditure. We apply a synthetic difference in differences analysis of the repeal of nursing home CON in Pennsylvania, Indiana, and North Dakota. Figure \ref{fig:nh_con_map} shows a complete picture of which states did and did not have nursing home CON regulations during our sample period. We find that the repeal of nursing home CON laws in Pennsylvania led to an increase in Medicaid expenditures of \$1.32 billion which represents a 70\% increase relative to its pre-treatment mean Medicaid expenditure. The repeal of nursing CON laws in Indiana led to an increase in Medicaid expenditures of \$334 million  which represents a 40\% increase relative to its pre-treatment mean Medicaid expenditure. The effect of North Dakota, however, are unreliable due to a clear lack of pre-treatment parallel trends. 

We find that this repeal caused an increase in the number of nursing home beds which is consistent with the predictions of our model. We also find that the repeal led to a substantial increase in total nursing homes, implying that facilities soon after the repeal were much newer which factors into the perception of nursing home quality that consumers have. The repeal also led to an increase in specialized care beds, which \citet{grabowski2010quality} show is correlated with higher quality of service. Finally, the effect of the repeal on total nursing home expenditures is nearly the same as the effect on nursing home expenditures which suggests that most, if not all, of the effect of repealing CON on expenditure is coming from its effect on Medicaid. We further analyze this hypothesis by testing if the repeal had any effect on Medicare and find that no evidence that it did. All results where we find evidence for parallel trends are consistent with the predictions of our model. These empirical results show that the effect of repealing nursing home CON is substantial and economically meaningful. Our model and empirical evidence supports the hypothesis that inter-state migration within the current Medicaid system incentivizes state politicians to impose nursing home CON restrictions that are higher than they would be in the absence of inter-state competition.

The paper is organized as follows: Section \ref{policy_background} overviews the policy background of CON laws and its literature with a focus on nursing homes; Section \ref{model} describes the model used to understand the impact of nursing home CON on access to nursing home services and nursing home expenditures; Section \ref{data} describes the data sources used; Section \ref{empirical_strategy} summarizes the synthetic difference in differences method and contrasts this empirical strategy with the traditional synthetic control and difference in differences methods; Section \ref{results} highlights the results of the synthetic difference-in-difference analysis and also contrasts it with the results from the synthetic control and difference in differences analyses; Section \ref{conclusion} offers a conclusion and the policy implications of this research.


\end{document}