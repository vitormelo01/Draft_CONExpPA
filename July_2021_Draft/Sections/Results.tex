\documentclass[../Main.tex]{subfiles}

\begin{document}

CON laws impose bureaucratic costs to nursing home providers by requiring those planning on opening a new nursing home or expanding a current facility to first show to a regulatory body that their region needs the service. In the context of a simple supply and demand model, these regulations would act to decrease the supply of nursing home services. Relative to an unregulated industry, therefore, we would expect to see a higher price, a lower quantity, and excess demand involving individuals willing to pay the unregulated price, but not the regulated price.\footnote{An alternative way to think about CON laws is that they impose a restriction on the quantity of nursing home services/beds. This restriction would result in a higher price, a lower quantity, and excess demand, the same outcome as modelling the regulations as a government-imposed decrease in supply.} It follows that when Pennsylvania dropped it's CON regulations, a resultant decrease in price and increase in quantity would be expected.\\ 
\indent When Pennsylvania dropped it's CON laws, this likely not only affected the incentives of producers and consumers in Pennsylvania, but also the incentives of individuals in surrounding states. The reason is that all states surrounding Pennsylvania maintained their CON regulations when Pennsylvania dropped theirs, and were therefore in a similar position with excess demand at the prices and quantities resulting from the regulation. As prices dropped and quantities of nursing home services/beds increased in Pennsylvania, it is likely that at least some elderly individuals in the surrounding states migrated to the new nursing home facilities. This increase in demand from surrounding states would act to increase prices and further increase quantities.\\ 
\indent Taking the direct effects in Pennsylvania of dropping their CON regulations, together with the indirect effects from surrounding states, we would expect to see an unambiguous increase in the quantity of nursing home services in Pennsylvania. Moreover, this increase in quantity is commensurate with the size of the demand increase from individuals in surrounding states. If the increase in demand from individuals in surrounding states is significant, this increase in quantity is expected to be quite sizable. The effect on government expenditure, however, which is a function of both price and quantity, is theoretically ambiguous. That notwithstanding, if there is a significant increase in demand from surrounding states, the resulting combination of higher prices and quantities would likely result in an increase in expenditure, particularly medicaid expenditure since medicaid, not medicare, covers long-term care. Furthermore, since medicaid is a joint federal and state program, individuals moving from surrounding states wishing to utilize medicaid to finance their nursing home services in Pennsylvania would need to change their residency status, and would therefore start receiving medicaid benefits from Pennsylvania. Thus, by dropping their CON laws, Pennsylvania ends up paying part of the bills for individuals from surrounding states, who otherwise.\\

\end{document}