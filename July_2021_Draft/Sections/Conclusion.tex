\documentclass[../Main.tex]{subfiles}

\begin{document}
Nursing homes are regulated by CON regulations more often than any other healthcare service. Nursing homes are often the most restrictive CON regulation, making it effectively impossible for healthcare providers to offer any additional beds in some states. This paper argues that interstate competition and the substantial effects of CON on nursing home Medicaid expenditures play an important role in the explaining why nursing homes are so heavily regulated. The model developed in this paper shows that interstate competition to avoid nursing home Medicaid costs creates incentives for state politicians to impose harsher nursing home CON restrictions than they would impose in the absence of such competition.

We support the predictions of our model by analyzing the effect of removing nursing home CON regulations on Medicaid nursing home expenditure, total nursing home expenditure, and the quantity of nursing home beds, nursing homes, and specialized care nursing home beds. Our primary results on Medicaid show that the repeal of nursing home CON laws in PA led to an increase in annual nursing home Medicaid expenditures of \$1.32 billion, which represents a 70\% increase relative to its pre-treatment mean expenditure. The same repeal in Indiana led to an increase in nursing home Medicaid expenditures of \$334 million per year, representing a 40\% increase relative to its pre-treatment mean expenditure. These results are not only consistent with the predictions of our model, but also show that nursing home CON has an enormous effect on Medicaid expenditure. We also find that repealing this regulation led to an increase in total nursing home expenditure and the quantity of nursing home beds, nursing homes, and specialized nursing home beds.

The interstate competitive pressures to regulate nursing homes and likely much of the effect of nursing home CON on Medicaid expenditure are an artifact of the current Medicaid system. When, for example, a nursing home consumer from New Jersey who is eligible for Medicaid moves to PA, they must update their residency as soon as they move. New Jersey would not pay for their Medicaid costs, but PA would. This only happens because the Medicaid system was designed in such a way. If nursing home consumers were not required to update their residency when moving to an out-of-state nursing home or if the state where this individual is from was otherwise required to cover the Medicaid expenses, the effect of CON on Medicaid expenditures and the pressure to regulate nursing homes through CON would decrease.  Thus, we propose a small but meaningful reform of the Medicaid system in a way that holds accountable the states where individuals are from for their Medicaid expenditure, even if they chose to go to a nursing home in a different state. Such reform would substantially reduce the costs of repealing nursing home CON regulations and thereby reduce the competitive pressure to regulate nursing homes. 

We conclude by stating what this study does not show. We do not show that repealing nursing home CON has a negative welfare effect on affected states. Such a complete welfare analysis would require one to estimate and quantify the quality effects of repealing this regulation which we cannot do due to a lack of data availability. Our focus in this paper is on the political pressures that could explain why nursing homes are so heavily regulated by CON laws. Future research should further explore how nursing home CON regulations affect the quality of services. Additionally, much of the current literature on CON is focused on the effects of the overall presence of CON regulations. Yet, the effects, prevalence, and restriction levels of CON laws vary by healthcare service. We focus on nursing homes due to its high prevalence and restrictiveness, but future research on the political economy and effects of other CON regulations (e.g. hospital beds, psychiatric services, ambulatory surgical centers, mental disability facilities, among others) could provide important policy relevant insights on healthcare regulation. 



\end{document}