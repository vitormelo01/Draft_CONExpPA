\documentclass[../Main.tex]{subfiles}

\begin{document}

Data on when states implemented and repealed Nursing Home CON come from the American Health Planning Association (AHPA) and the National Conference of State Legislatures. The data from the AHPA were compiled by \citet{stratmann2014certificate}. The information from these data sources were verified through each state’s statutes and were found to be correct in all but two instances (Connecticut and Wisconsin). The data on these two instances were used based on the information in their statutes. These data are used to create a binary variable Nursing Home CON which represents whether a state has Nursing Home CON laws at a given time period.\\
\indent The main dependent variable used in this paper measures annual per capita total nursing home expenditure and Medicaid nursing home expenditure in each state. These data come from the National Health Expenditure Accounts (NHEA) and are available from 1980-2014. These variables are adjusted for inflation using the Consumer Price Index and all dollar amounts are reported in 2015 dollars. In all our analyses, these variables are also divided by state population (US Census) to offer real per capita measures. Data on the quantity of nursing homes and nursing home beds are from the Centers for Medicare and Medicaid Services’ (CMS) Provider of Services file. The Provider of Services file includes data collected by CMS regional offices of all health care providers and distinguishes providers by type, name, and address of each facility. These data were adjusted for population and collapsed into the variables Quantity of Nursing Homes, which represents the number of facilities classified by CMS as a Skilled Nursing Facility per 100,000 citizens in each given states, and Nursing Home Beds, which represents the quantity of beds in these facilities per 100,000 citizens. These data are available from 1991-2014. CMS also has two types of nursing home/assistance living categories that represents facilities with some nursing home beds and some assistance living beds. These categories are not used for this study since the ratio of assistance living beds per nursing home bed in these facilities is unknown. \\
\indent Time-varying covariates used in our analyses come from various sources and are available from 1980-2014. Our state-year demographic characteristics, including the proportion of married individuals, the proportion of white individuals, the proportion of individuals with a bachelor degree or higher, and shares of individuals aged 25 to 44, 45-64, and over 65 come from the Integrated Public Use Microdata Series (IPUMS) compilation of the Current Population Survey. Data on Income per capita and population come from the U.S. Bureau of Economic Analysis. Income per capita is adjusted for inflation using the Consumer Price Index and is reported in 2015 dollars. The state unemployment rate data come from the Bureau of Labor Statistics’ Local Area Unemployment Statistics. Data on top 1\% income shares come from the Frank-Sommeiller-Price series \citep{frank2015performance} and Gini Coefficients come from the U.S. Census Bureau.

\end{document}