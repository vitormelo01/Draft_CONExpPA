\documentclass[../Main.tex]{subfiles}

\begin{document}

To empirically test the predictions of our model, we pull from a variety of data sources. Data on when states implemented and repealed Nursing Home CON come from the American Health Planning Association (AHPA) and the National Conference of State Legislatures. The nursing home CON status for each state was consistent across the two sources in all but two cases (Connecticut and Wisconsin). The information from these data sources were verified through each state’s statutes and were found to be correct in all but two instances (Connecticut and Wisconsin). The data on these two instances were used based on the information in their statutes. These data are used to create a binary variable Nursing Home CON which represents whether a state has Nursing Home CON laws at a given time period.\\
\indent The main expenditure dependent variables used in this paper measure annual per capita Medicaid and total government nursing home expenditure in each state. These data come from the National Health Expenditure Accounts (NHEA) and are available from 1980-2014. These variables are adjusted for inflation using the Consumer Price Index and all dollar amounts are reported in 2015 dollars. In all our analyses, these variables are also divided by state population (US Census) to offer real per capita measures.\\
\indent Our dependent variable measures on the quantity and quality of nursing home services come from the Centers for Medicare and Medicaid Services’ (CMS) Provider of Services file. The Provider of Services file includes data collected by CMS regional offices of all health care providers and distinguishes providers by type, name, and address of each facility. These data were adjusted for population and collapsed into variables measuring the quantity of nursing homes per 100,000, which represents the number of facilities classified by CMS as Skilled Nursing Facilities per 100,000 citizens in each given states; the quantity of nursing home beds per 100,000, which represents the quantity of beds in these facilities per 100,000 citizens; and the quantity of specialized beds per 100,000, which represents the quantity of specialized care beds per 100,000 citizens. Each of these variables provide direct measures of the quantity of nursing home services provided. We use the quantity of specialized beds per 100,000 as our main measure of quality, since \citet{grabowski2010quality} show that is is correlated with higher quality of service in general. Moreover, to the extent that newer beds and facilities are of relatively higher quality, increases in these measures indicate a higher quality of services, all else equal. These quantity and quality measures from the CMS Provider of Services file are available from 1991-2014.\footnote{CMS also has two types of nursing home/assistance living categories that represents facilities with some nursing home beds and some assistance living beds. These categories are not used for this study since the ratio of assistance living beds per nursing home bed in these facilities is unknown.} \\
\indent Time-varying covariates used in our analyses come from various sources and are available from 1980-2014. Our state-year demographic characteristics, including the proportion of married individuals, the proportion of white individuals, the proportion of individuals with a bachelor degree or higher, and shares of individuals aged 25 to 44, 45 to 64, and over 65 come from the Integrated Public Use Microdata Series (IPUMS) compilation of the Current Population Survey. Data on Income per capita and population come from the U.S. Bureau of Economic Analysis. Income per capita is adjusted for inflation using the Consumer Price Index and is reported in 2015 dollars. State unemployment rate data come from the Bureau of Labor Statistics’ Local Area Unemployment Statistics. Data on top 1\% income shares come from the Frank-Sommeiller-Price series \citep{frank2015performance} and Gini Coefficients come from the U.S. Census Bureau.

\end{document}