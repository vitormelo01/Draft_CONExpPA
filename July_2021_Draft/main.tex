\documentclass[12pt]{article}
\usepackage{amsmath}
\usepackage[margin=1in]{geometry}
\usepackage{graphicx}
\usepackage{epstopdf}
\usepackage{setspace}
\usepackage{amsthm,mathpazo}
\usepackage{tikz}
\usepackage{pgfplots}
\usepackage{verbatim}
\usepackage{caption}
\usepackage[round]{natbib}
%\usepackage{biblatex}
\usepackage{filecontents}
\usepackage{enumitem}
\usepackage{hyperref}
\bibliographystyle{ecta}
\usepackage{comment}
%\usepackage[labelsep=period]{caption}
%\captionsetup[table]{name=TABLE}
\usepackage{setspace,graphicx,epstopdf,amsmath,amsfonts,amssymb,amsthm}
\usepackage{mathtools}
\usepackage{dcolumn}
\usepackage{pdflscape}
\usepackage{lscape}
\usepackage{rotating}
\usepackage{xcolor}
\usepackage{marginnote,enumitem,rotating,fancyvrb}
\usepackage{hyperref,float}
\newtheorem{Proposition}{Proposition}
\usepackage{subcaption}
\usepackage{caption}
\usepackage{pgfplots}
\usepackage{pgfplotstable}
\pgfplotsset{compat=newest}
\pgfplotsset{compat=newest}
\usetikzlibrary{patterns}
%\pgfkeys{/pgf/number format/.cd,fixed,'.
%      use period}
\usepackage{tikz}
\usetikzlibrary{patterns,intersections}
\usetikzlibrary{arrows,calc}
\tikzset{
%Define standard arrow tip
>=stealth',
%Define style for different line styles
help lines/.style={dashed, thick},
axis/.style={<->},
} 

\usepackage[linesnumbered,ruled]{algorithm2e}
\DeclareMathOperator*{\argmin}{arg\,min}

\usepackage{xcolor}
\hypersetup{
    colorlinks=true,
    linkcolor={red!50!black},
    citecolor={blue!50!black},
    filecolor=blue,      
    urlcolor={blue!80!black}
}
\urlstyle{same}


\renewcommand*\thetable{\arabic{table}}
%\renewcommand{\thetable}{\Roman{table}}
\renewcommand*\thefigure{\arabic{figure}}

%\renewcommand{\familydefault}{\rmdefault}
\usepackage{mathptmx}   %Times New Roman Font
%\linespread{1.25}   %1.5 spacing
\usepackage{xr}     %Allows for referencing across docs
\externaldocument{Appendices}

\usepackage{subfiles}

\doublespacing

\graphicspath{{Figures_and_Tables/}}

\begin{document}

\setlist{noitemsep}  % Reduce space between list items (itemize, enumerate, etc.)

\title{\textsc{Why Are Nursing Homes so Heavily Regulated? The Effects and Political Economy of Certificate-of-Need Laws}\thanks{We thank participants of the Public Economics Workshop at Clemson University and the Southern Economic Conference 2021 for helpful comments. Melo would like to thank the support from the Initiative on Enabling Choice & Competition at University of Chicago and the Hayek Center at Clemson University. All remaining errors are ours.}\\
	$~$\\}

\medskip

\author{\textbf{Vitor Melo\protect\thanks{Vitor Melo is a Ph.D. Candidate at Clemson University in the John E. Walker Department of Economics. Email: vmelo@clemson.edu.}} \\ Clemson University
\and
\textbf{Elijah Neilson\protect\thanks{Elijah Neilson is an Assistant Professor of Economics at Southern Utah University in the Dixie L. Leavitt School of Business. Email: elijahneilson@suu.edu.}} \\ Southern Utah University
  	}		
	%\date{October 27, 2017.}

\date{}              % No date for final submission

% Create title page with no page number




\renewcommand{\thefootnote}{\fnsymbol{footnote}}

\singlespacing

\maketitle

%\vspace{-.2in}
\begin{abstract}
\noindent 
Certificate-of-Need (CON) laws restrict the quantity of health services by requiring providers to convince a government board that there is a ``need" for a given service in their region. Nursing home CON regulations are the most prevalent and often the most restrictive CON regulation. This paper explores why this regulation is so widespread by developing a model where firms compete for a mobile nursing home consumer. The model shows that repealing CON leads to an increase in the quality and quantity of nursing home services, as well as expenditures. The model also predicts that inter-state competition incentivizes state politicians to impose nursing home CON restrictions that are higher than they otherwise would be. We support these predictions by analyzing the effect of this regulation on Medicaid and total expenditures, as well as on the number of nursing home beds, nursing home facilities, and specialized beds. Applying a synthetic difference in differences analysis, we find that the repeal of nursing home CON laws led to an increase in Medicaid expenditures of \$1.32 billion and \$334 million in Pennsylvania and Indiana, respectively, while the estimated effect in North Dakota is not reliable due to a lack of evidence for parallel trends. We also find that this repeal caused an increase in the number of nursing home beds, nursing home facilities, specialized beds, and total expenditure.

\bigskip
			
			\noindent\emph{JEL Classification: I11, I14, L51} 
			
			\bigskip
			
			\noindent\emph{Keywords: Nursing Homes, Healthcare  Regulation, Certificate-of-Need Laws. } 
\end{abstract}

	\bibliographystyle{aer}
	
	\maketitle

\medskip

\thispagestyle{empty}

\clearpage

\onehalfspacing
\setcounter{footnote}{0}
\renewcommand{\thefootnote}{\arabic{footnote}}
\setcounter{page}{1}

\doublespacing
\setcounter{footnote}{0}
\renewcommand{\thefootnote}{\arabic{footnote}}
\setcounter{page}{1}

\doublespacing

\section{Introduction} \label{introduction}
\subfile{Sections/Introduction}

\section{Policy Background and Literature Review} \label{policy_background}
\subfile{Sections/Policy_Background}

\section{Model} \label{model}
\subfile{Sections/Model}

\section{Data} \label{data}
\subfile{Sections/Data}

\section{Empirical Strategy} \label{empirical_strategy}
\subfile{Sections/Empirical_Strategy_SDID}

\section{Results} \label{results}
\subfile{Sections/Results_SDID}

\section{Conclusion} \label{conclusion}
\subfile{Sections/Conclusion}


\clearpage

% Bibliography.
\singlespacing
\bibliographystyle{ecta}
\bibliography{References}

\subfile{Sections/Tables_SDID_NoBord_NoCov}

\newpage 
\subfile{Sections/Figures_SDID_NoBord_NoCov}

\end{document}

