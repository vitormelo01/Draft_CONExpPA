\documentclass[12pt]{article}
\usepackage{amsmath}
\usepackage[margin=1in]{geometry}
\usepackage{graphicx}
\usepackage{epstopdf}
\usepackage{setspace}
\usepackage{amsthm,mathpazo}
\usepackage{tikz}
\usepackage{pgfplots}
\usepackage{verbatim}
\usepackage{caption}
\usepackage[round]{natbib}
%\usepackage{biblatex}
\usepackage{filecontents}
\usepackage{enumitem}
\usepackage{hyperref}
\bibliographystyle{ecta}
\usepackage{comment}
%\usepackage[labelsep=period]{caption}
%\captionsetup[table]{name=TABLE}
\usepackage{setspace,graphicx,epstopdf,amsmath,amsfonts,amssymb,amsthm}
\usepackage{mathtools}
\usepackage{dcolumn}
\usepackage{pdflscape}
\usepackage{lscape}
\usepackage{rotating}
\usepackage{xcolor}
\usepackage{marginnote,enumitem,rotating,fancyvrb}
\usepackage{hyperref,float}
\newtheorem{Proposition}{Proposition}
\usepackage{subcaption}
\usepackage{caption}
\usepackage{pgfplots}
\usepackage{pgfplotstable}
\pgfplotsset{compat=newest}
\pgfplotsset{compat=newest}
\usetikzlibrary{patterns}
%\pgfkeys{/pgf/number format/.cd,fixed,'.
%      use period}
\usepackage{tikz}
\usetikzlibrary{patterns,intersections}
\usetikzlibrary{arrows,calc}
\tikzset{
%Define standard arrow tip
>=stealth',
%Define style for different line styles
help lines/.style={dashed, thick},
axis/.style={<->},
} 

\usepackage{lineno}
\linenumbers

\usepackage[linesnumbered,ruled]{algorithm2e}
\DeclareMathOperator*{\argmin}{arg\,min}

\usepackage{xcolor}
\hypersetup{
    colorlinks=true,
    linkcolor={red!50!black},
    citecolor={blue!50!black},
    filecolor=blue,      
    urlcolor={blue!80!black}
}
\urlstyle{same}


\renewcommand*\thetable{\arabic{table}}
%\renewcommand{\thetable}{\Roman{table}}
\renewcommand*\thefigure{\arabic{figure}}

%\renewcommand{\familydefault}{\rmdefault}
\usepackage{mathptmx}   %Times New Roman Font
%\linespread{1.25}   %1.5 spacing
\usepackage{xr}     %Allows for referencing across docs
\externaldocument{Appendices}

\usepackage{subfiles}

\doublespacing

\graphicspath{{Figures_and_Tables/}}

\begin{document}

\setlist{noitemsep}  % Reduce space between list items (itemize, enumerate, etc.)

\title{\textsc{Why Are Nursing Homes so Heavily Regulated? The Effects and Political Economy of Certificate-of-Need Laws}%\thanks{We thank Casey Mulligan, Tom Phillipson, Andy Hanssen, participants of the Conference of the Initiative on Enabling Choice and Competition at the University of Chicago 2022, the Public Choice Society Conference 2022, the Southern Economic Conference 2021, and the Public Economics Workshop at Clemson University for helpful comments. Melo would like to thank the support from the Initiative on Enabling Choice & Competition at University of Chicago and the Hayek Center at Clemson University. All remaining errors are ours.}\\
	$~$\\}

\medskip

%\author{\textbf{Vitor Melo\protect\thanks{Vitor Melo is a Ph.D. Candidate at Clemson University in the John E. Walker Department of Economics. Email: vmelo@clemson.edu.}} \\ Clemson University
%\and
%\textbf{Elijah Neilson\protect\thanks{Elijah Neilson is an Assistant Professor of Economics at Southern Utah University in the Dixie L. Leavitt School of Business. Email: elijahneilson@suu.edu.}} \\ Southern Utah University
  	%}		
	%\date{October 27, 2017.}

\date{}              % No date for final submission

% Create title page with no page number




\renewcommand{\thefootnote}{\fnsymbol{footnote}}

\singlespacing

\maketitle

%\vspace{-.2in}
\begin{abstract}
\noindent 
Certificate-of-Need (CON) laws restrict the quantity of health services by requiring that providers convince a government board that there is a ``need'' for a given service in their region. Nursing home CON laws have historically been the most prevalent and often most restrictive CON law. This paper explores why these nursing home regulations are so widespread by building on previous models of interstate competition. Our model predicts that repealing these laws will increase Medicaid expenditures, as well as the quality and quantity of nursing home services. We assess the predictions of our model by analyzing the repeal of nursing home CON laws in Pennsylvania, Indiana, and North Dakota. Using the synthetic difference-in-differences method, we find that repealing these regulations increased per capita nursing home Medicaid expenditures by \$102 (70\%) and \$49 (38\%) in Pennsylvania and Indiana, respectively. We also find that this repeal caused an increase in the quantity of nursing home beds, nursing home facilities, and specialized care beds, most notably in Pennsylvania. Our findings and theoretical framework are consistent with the proposition that interstate competition incentivizes politicians to impose more restrictive nursing home CON laws than they would in the absence of such competition. We argue that such competitive pressures are an artifact of Medicaid and offer alternatives to the current system. 







%Certificate-of-Need (CON) laws restrict the quantity of health services by requiring that providers convince a government board that there is a ``need'' for a given service in their region. Nursing home CON laws have historically been the most prevalent and often most restrictive CON law. This paper explores why nursing home CON laws are so widespread and the effect they have on Medicaid expenditure and the quantity and quality of nursing home services. We develop a model in which repealing nursing home CON regulations is predicted to increase both the quality and quantity of nursing home services, as well as Medicaid expenditures. We assess the predictions of our model by analyzing the repeal of nursing home CON laws in Pennsylvania, Indiana, and North Dakota. Using the synthetic difference-in-differences method, we find that repealing these regulations caused an increase in per capita nursing home Medicaid expenditures of \$102 (70\%) and \$49 (38\%) in Pennsylvania and Indiana, respectively. We also find that this repeal caused an increase in the quantity of nursing home beds, nursing home facilities, and specialized care beds, most notably in Pennsylvania. Our findings and theoretical framework are consistent with the proposition that inter-state competition incentivizes politicians to impose more restrictive regulations than they would in the absence of such competition.

%Certificate-of-Need (CON) laws restrict the quantity of health services by requiring that providers convince a government board that there is a ``need'' for a given service in their region. Nursing home CON regulations have been the most prevalent and often the most restrictive CON regulation. This paper estimates the causal effects of nursing home CON and explores why this regulation is so widespread. Our model shows that repealing nursing home CON regulations leads to an increase in the quality and quantity of nursing home services, as well as Medicaid expenditures. The model also predicts that inter-state competition incentivizes state politicians to impose nursing home CON restrictions that are higher than they otherwise would be. Applying a synthetic difference in differences analysis, we find that the repeal of nursing home CON laws led to an increase in nursing home Medicaid expenditures of 70\% (\$1.32 billion) and 38\% (\$334 million) in Pennsylvania and Indiana, respectively. We also find that this repeal caused an increase in the number of nursing home beds, nursing home facilities, specialized beds, and total expenditure.

% , while the estimated effect in North Dakota is not reliable due to a lack of evidence for parallel trends.
\bigskip
			
			\noindent\emph{JEL Classification: I11, I14, L51} 
			
			\bigskip
			
			\noindent\emph{Keywords: Nursing Homes, Healthcare  Regulation, Certificate-of-Need Laws. } 
\end{abstract}

	\bibliographystyle{aer}
	
	\maketitle

\medskip

\thispagestyle{empty}

\clearpage

\onehalfspacing
\setcounter{footnote}{0}
\renewcommand{\thefootnote}{\arabic{footnote}}
\setcounter{page}{1}

\doublespacing
\setcounter{footnote}{0}
\renewcommand{\thefootnote}{\arabic{footnote}}
\setcounter{page}{1}

\doublespacing

\section{Introduction} \label{introduction}
\subfile{Sections/Introduction}

%\section{Policy Background} \label{policy_background}
%\subfile{Sections/Policy_Background}

\section{Theoretical Model} \label{model}
\subfile{Sections/Model}

\section{Data and Background} \label{data}
\subfile{Sections/Data}

\section{Empirical Strategy} \label{empirical_strategy}
\subfile{Sections/Empirical_Strategy_SDID}

\section{Results} \label{results}
\subfile{Sections/Results_SDID}

\section{Conclusion} \label{conclusion}
\subfile{Sections/Conclusion}


\clearpage

% Bibliography.
\singlespacing
\bibliographystyle{ecta}
\bibliography{References}

\subfile{Sections/Tables_SDID_NoBord_NoCov}

\newpage 
\subfile{Sections/Figures_SDID_NoBord_NoCov}

\end{document}

